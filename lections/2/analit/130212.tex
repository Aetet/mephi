\documentclass[12pt]{article}
%\usepackage{ucs}
\usepackage[utf8x]{inputenc} % Включаем поддержку UTF8
\usepackage[russian]{babel}  % Включаем пакет для поддержки русского языка
\usepackage{amsmath}
\usepackage{amssymb}

\title{Аналитическая геометрия и линейная алгебра}

\date{}
\author{abcdw}

\begin{document}
    \maketitle
    A, B - матрицы $n \times n$. \newline
    $|A * B| = |B * A| = |A| * |B|$ \newline
    $B = \begin{pmatrix}
        \vec b_1 \\
        \vec b_2 \\
        \vdots \\
        \vec b_n
    \end{pmatrix}$
    $\tilde B = \begin{pmatrix}
        \vec b_{i_1} \\
        \vec b_{i_2} \\
        \vdots \\
        \vec b_{i_n}
    \end{pmatrix}$
    $det \tilde B = (-1)^{t(i_1, i_2, \dots, i_n)} det B$ \newline
    много добавить. $\leftarrow$

    Обратная матрица. \newline

    A - матрица $n \times n$. \newline
    B - обратная к матрица A, если \newline
    A * B = B * A = E \newline
    E - единичная. \newline
    B - матрица $n \times n$ \newline

    A - называется невырожденной, если $|A| \not = 0$ \newline

    Теорема. \newline
    A - обратима $\Leftrightarrow |A| \not = 0$ \newline
    В одну сторону: \newline
    $|A * B| = |E| = 1$ \newline
    $|A| * |B| = |A * B| \Rightarrow |A| \not = 0$ \newline
    В другую: \newline
    $B = \frac1{\triangle}\begin{pmatrix}
        A_{11} & A_{21} & \dots & A_{n1} \\
        A_{12} & A_{22} & \dots & A_{n2} \\
        \vdots & \vdots & \ddots & \vdots \\
        A_{1n} & A_{2n} & \dots & A_{nn}
    \end{pmatrix} = \frac1{\triangle}C
    $

    C - транспонированная матрица из алгебраических дополнений. $C~=~(A_{ij})^T$ \newline
    Проверим. \footnote{дописать} \newline

    Теорема. \newline
    Если существует обратная матрица, то она единственна. \newline
    $B_1 = B_1 * E = B_1 * (A * B_2) = (B_1 * A) * B_2 = B_2$ \newline

    Пример. \newline
    $A = \begin{pmatrix}
        2 & 5 & 7 \\
        6 & 3 & 4 \\
        5 &-2 &-3
    \end{pmatrix}
    A^{-1} = \frac1{-1} \begin{pmatrix}
        -1 & 1 & -1 \\
        38 & -41 & 34 \\
        -27 & 29 & -24
    \end{pmatrix}
    $

    Свойства. \newline
    1. $(A^{-1})^{-1} = A$ \newline
    2. $(A * B)^{-1} = B^{-1} * A^{-1}$ \newline
    3. $(A^T)^{-1} = (A^{-1})^T$ \newline

    Линейная зависисмость и независимость столбцов и строк матрицы. \newline
    $\vec a_i$ - строки. \newline
    $a_{\downarrow i}$ - столбцы. \newline

    Система столцов называется линейно зависимой, если $\exists \alpha_i \not = 0: \alpha_1 a_{\downarrow 1} + \alpha_2 a_{\downarrow 2} + \dots + \alpha_k a_{\downarrow k} = 0$ \newline
    Столбец $a_{\downarrow}$ является линейной комбинацией, если $\exists \alpha_i$. \newline
    Столбцы ЛЗ $\Leftrightarrow$ один линейно выражается через остальные. \newline
    Доказательство. \newline

    Достаточное условие линейной зависимости столбцов. \newline
    1. Если в системе имеется нулевой столбец. \newline
    2. Если в системе столбцов имеется линейная подсистема. Следствие: если система ЛНЗ, то и любая подсистема ЛНЗ. \newline

    Ранг матрицы. \newline

    
\end{document}
