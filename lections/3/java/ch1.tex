\chapter{Введение}
\section{Офигительные сайты}

\begin{enumerate}
    \item \href{http://jcp.org}{java community portal}
\end{enumerate}


\subsection{Офигительные книги}
\begin{enumerate}
  \item Хорстман Java 2
  \item Шилдт Java
  \item Joshua Block Effective Java
  \item Oracle Java Proffessional
  \item Design Pattern's
  \item javadoc
\end{enumerate}

\section{Язык программирования}
\begin{definition}
  Язык программирования - средство описания вычислений для людей и машин.
\end{definition}

\subsection{По поколениям}
\begin{description}
  \item[1 поколение] машинные языки
  \item[2 поколение] Assembler
  \item[3 поколение] C, C++, Java
  \item[4 поколение] SQL
  \item[5 поколение] Lisp, Prolog, Haskell
\end{description}

\subsection{По парадигмам}
\begin{enumerate}
  \item машинные
  \item assembler
  \item процедурные
  \item объектно-ориентированные
  \item функциональные
  \item логические
\end{enumerate}

\section{ООП(объектно-ориентированное программирование)}
Проблемы процедурного программирования
\begin{enumerate}
  \item конфликты именования
  \item отделение кода от данных
  \item не соответсвует образу мышления
\end{enumerate}

2 способа определения множества(абстрагирование от конкретных предметов)
\begin{enumerate}
  \item перечисление
  \item выделение характерных свойств объекта
\end{enumerate}

\begin{definition}
  Аггрегация - объект может содержать в себе другие объекты.
\end{definition}

\begin{definition}
  Интерфейс - набор правил видимых извне объекта
\end{definition}

\subsection{Постулаты ООП}

\begin{description}
  \item[Абстракция] мыслить более широкими понятиями
  \item[Инкапсуляция] объеденение данных и кода, котоый ими оперирует (!)
  \item[Наследование] наследование свойств
  \item[Полиморфизм]
\end{description}
