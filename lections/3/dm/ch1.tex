\chapter{Основные понятия}
\section{Введение}

\begin{enumerate}
  \item машина Тьюринга
  \item синтаксическая переработка слов Маркова
  \item рекурсивные функции
\end{enumerate}

\begin{definition}
  Алгоритм - точное предписание о выполнении в некотором порядке системы
  операций, определяющих процесс перехода от исходных данных к искомому
  результату для решения задачи данного типа.
\end{definition}

Свойства алгоритма.

\begin{description}
  \item[Определенность] общепринятость и точность
  \item[Массовость]
  \item[Результативность]
  \item[Элементарность шагов]
\end{description}


\begin{enumerate}
  \item Множество допустимых исходных данных
  \item Начальное состояние
  \item Промежуточные состояния
  \item Правила перехода от одних данных к другим
  \item Множество результатов
\end{enumerate}

\section{Машины Тьюринга и Поста}
\begin{enumerate}
  \item Маленькие шаги
  \item Устройство должны быть универсальным
  \item Механизм должен быть максимально простым
\end{enumerate}

\begin{definition}
  Машина Тьюринга - абстрактная ``вычислтельная машина'' некоторого точно
  охарактеризованного типа, дающая пригодное для целей математического 
  рассмотрения уточнение общего интуитивного представления об алгоритме.
\end{definition}

Требования к машине Тьюринга
\begin{enumerate}
  \item Машина должна быть детерминирована
  \item Машина должна решать класс задач
  \item Результат должен быть ``читаемым''
\end{enumerate}

\subsection{Одноленточная машина тьюринга}

\begin{definition}
  Машина тьюринга - кибернетическое устройство, содержащее:
  \begin{enumerate}
    \item Бесконечная лента, разделенная на ячейки
    \item Управляющая головка, способная читать символы, содержащиейся 
      в ячейке ленты и записывать в эти ячейки
    \item Выделенная ячейка памяти, содержащаясимвол внутреннего алфавита, 
      задающий состояния машины Тьюринга
    \item Механическое устройств для перемещания головки
    \item Функциональные схемы(программа)
  \end{enumerate}
\end{definition}

\begin{remark}
  $\lambda$ - пустой символ
\end{remark}

\begin{remark}
  Сделаем одностороннюю бесконечную ленту.
\end{remark}
\begin{proof}
  Просто пронумеруем ячейки $\dots 5, 3, 1, 0, 2, 4, 6 \dots$
\end{proof}

\begin{remark}
  %$\etta$ - первый служебный символ
\end{remark}

\begin{definition}
  Управляющая головка - это некоторое устройство, которое может перемещаться
  вдоль ленты так, что в каждый рассматриваемый момент времени оно находится
  напротив определенной ячейки ленты.
\end{definition}

\begin{definition}
  Внутренняя память машины - это выделенная ячейка памяти, которая в каждый 
  рассматреваемый момент времени находится в некотором ``состоянии''.
\end{definition}

Внутренний алфавит:
\begin{remark}
  $S_0$ - начальное состояние.
\end{remark}

\begin{remark}
  $A, B, C, A_0, B^*$ - промежуточное состояние.
\end{remark}

\begin{remark}
  $\Omega$ - конечное состояние.
\end{remark}

\subsection{Механическое устройство}
\begin{enumerate}
  \item изменять состояние внутренней память
  \item одновременно изменить состояние ячейки
  \item сдвинуть головку влево или вправо
\end{enumerate}

\begin{definition}
  Конфигурация машиниы тьюринга - совокупность, образованная содержимым
  текущей обозреваемой ячейки $a_j$ и состоянием внутренней памяти $S_i$.
\end{definition}
\begin{example}
  $S_ia_j \to a_yRS_q$ \\
  конфигурация $\to$ действие
\end{example}

\subsection{Программа машины Тьюринга}
\begin{definition}
  Набор команд установленного формата.
\end{definition}

\begin{example}
  n состояний $S_i$ \\
  m символов $a_j$ \\
  $n \times m$ конфигураций
\end{example}

\begin{definition}
  Тезис тьюринга - любой алгоритм можно преобразовать в машину Тьбринга.
\end{definition}

\begin{example}
  \{a\} \\
  Написать 1 или 2 в зависимости от ``четности'' слова. \\
\end{example}
