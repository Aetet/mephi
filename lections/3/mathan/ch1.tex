\chapter{Числовые ряды}
\section{Определение}

\begin{definition}
  $U_1 + U_2 + U_3 + \dots = \sum\limits_{k = 0}^{\infty}{U_k}$
\end{definition}

\begin{definition}
  $S_n = $ - Частичная сумма
\end{definition}
\begin{definition}
  Ряд сходится, если $\exists \lim_{n \to \infty} \sum\limits_0^\infty U_k$
\end{definition}

\begin{definition}
  $\{a_n\} a_n = a_0 + \sum\limits_1^n (a_k - a_{k-1}$
\end{definition}

\begin{theorem}[Критерий Коши]
  Ряд сходится $ \Leftrightarrow \forall \varepsilon \exists N = N(\varepsilon)
  \forall n \geq N, \forall p |\sum\limits_{k=n+1}^{n+p} n_k| = |U_{n+1} + \dots
  + U_{n+p} = |S_{n+p} + S_{n}| < \varepsilon$
\end{theorem}

\begin{proof}
  $\forall \varepsilon > 0 \exists N: \forall n \geq N, \forall p
  |S_{n+p} + S_{n}| < \varepsilon$
\end{proof}

\begin{definition}
  Краевые условия
  Если ряд
\end{definition}

\begin{example}
  $\sum\limits_0^\infty z^n$
  $S_n(z) = \sum\limits_0^n z_n = \frac{1-z^{n+1}}{1-z}$
  При
\end{example}

\section{Действия с рядами}
\begin{theorem}
  Ряды $\sum U_k$ и $\sum V_k$ сходятся, тогда
  $\sum\alpha U_k = \alpha\sum U_k$\\
  $\sum U_k \pm V_k = \sum U_k \pm \sum V_k$
\end{theorem}
\begin{proof}
  $\sum\limits_{k=0}^\infty \alpha U_k = \lim_{n \to \infty}
  \sum\limits_0^n \alpha U_k = \alpha \lim_{n \to \infty} \sum\limits_0^n U_k
  = \alpha \sum\limits_0^\infty U_k$
\end{proof}

\begin{proof}
  Аналогично второе.
\end{proof}

\begin{remark}
  Сумма сходится $\not \rightarrow$ по отдельности. \\
  Еще свойство
  Нельзя раскрывать скобки и переставлять.
\end{remark}

\section{Ряды с неотрицательными членами}

$S_n$ - не строго возрастающая
Сходимость ряда эквивалентна ограниченности $S_n$

\begin{theorem}
  \begin{enumerate}
    \item $U_k \geq 0, V_k \geq 0 \forall k$
      Если $0 \leq U_k \leq V_k$, то $\sum V_k$ сходится $\Rightarrow$
      $\sum U_k$ сходится
      $\sum U_k$ расходится $\Rightarrow$ расходится $\sum V_k$ \\
    \item Если $\lim_{n\to \infty} \frac{U_k}{V_k} = A > 0$, то ряды
      сходятся или расходятся.
  \end{enumerate}
\end{theorem}

\begin{proof}
  Тут доказательство
\end{proof}

\begin{remark}
  Вместо существования предела достаточно предположить, что существуют такие числа
  p и q > 0 такие что $0 < q < \frac{U_k}{V_k} < p \forall k$
\end{remark}

\begin{theorem}[Признак Даламбера]
  Признаки\\
  \begin{enumerate}
    \item Если $\exists q \forall k \frac{U_{k+1}}{U_k} < q < 1$ сходится
    \item Если предел
  \end{enumerate}
\end{theorem}

\begin{proof}
  \begin{enumerate}
    \item Идея докозательства - сравнение с геометрической прогрессией.
    \item Для предельного случая
  \end{enumerate}
\end{proof}

\begin{theorem}[Признак Коши]
  $\sum U_k, U_k \geq 0$ \\
  \begin{enumerate}
    \item Если $\exists q < 1$, то $\forall k \sqrt[k]U_k \leq q < 1$
    \item Если $\exists \lim_{k \to \infty} \sqrt[k]U_k = q(\geq 0)$
  \end{enumerate}
  $q < 1$ - сходится \\
  $q > 1$ - расходится \\
  $q = 1$ - нужны дополнительные исследования\\
\end{theorem}
\begin{remark}
  $\overline \lim$ вместо $\lim$
\end{remark}

\begin{proof}
  Сравнение с геометрической прогрессией \\
  Если $\forall k \sqrt[k]U_k \leq q < 1 \Leftrightarrow U_k \leq q^k$
\end{proof}

\begin{definition}
  $\{a_n\}$ \\
  $\overline \lim_{n\to \infty} a_n$
\end{definition}

\begin{proof}[Признак Коши с верзним пределом]
  $\overline \lim_{k \to \infty} \sqrt[k]U_k = q < 1$
\end{proof}

\begin{remark}
  Признак Даламбера слабее признака Коши
\end{remark}
