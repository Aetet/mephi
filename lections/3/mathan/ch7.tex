\section{Поверхности в $\mathbb{R}^3$}
Пусть $\mathbb{D} \subset \mathbb{R}^2, \ \overline{\mathbb{D}}$ --- ее
замыкание.

\begin{definition}
  \label{def711}
  $S$ --- образ непрерывного отображения $x = x(u, v), y = y(u, v), \\ z = z(u,
  v)$ замкнутой области $\overline{\mathbb{D}} \subset \mathbb{R}_{uv}^2$ в
  пространство $\mathbb{R}_{xyz}^3$ называется непрерывной поверхностью.
\end{definition}
Само отображение называется параметрическим представлением поверхности.
Переменные $u, v$ --- параметры поверхности $S$.
\begin{gather}
  S = \{\pvec r(u, v) : (u, v) \in \overline{\mathbb{D}} \}
  \label{ch71:eq1}
\end{gather}
Вектор $\pvec r(u, v)$ --- где $\pvec r$ --- радиус вектор в $\mathbb{R}^3$ с
начала в начале координат и концом в точке($x(u, v), y(u, v), z(u, v)$).
Представление \eqref{ch71:eq1} назовем векторным представлением поверхности
$S$. Отметим, что одна и та же поверхность $S$ может иметь различные
параметрические представления.

\begin{definition}
  Поверхность $S$ назовем простой, если на ней нет кратных точек, то есть для
  любой точки $M \in S$ отображается лишь одна точка $(u,v) \in
  \overline{\mathbb{D}}$.
\end{definition}

\begin{definition}
  Если за параметры в одной из представленных поверхностей $S$ можно взять
  какие-либо две координаты пространства $\mathbb{R}^3$, то такое представление
  называется явным. Например, таковым является представление:
  \begin{gather*}
    S : x = x, y = y, z = \varphi(x, y) : (x, y) \in \overline{\mathbb{D}}
  \end{gather*}
  значит $S$ --- график $z = \varphi(x, y): (x, y) \in \overline{\mathbb{D}}$.
  Очевидно, что поверхность, допускающая явное представление, не имеет кратных
  точек.
\end{definition}

\begin{definition}
  Если в определении \eqref{def711} отображение является непрерывно
  диффиренцируемым, то $S$ называется непрерывно диффиренцируемой поверхностью.
\end{definition}

\begin{example}
  $x = R \cos \varphi \cos \psi, y = R \sin \varphi \cos \psi, z = R \sin
  \psi$, где $0 \leq \varphi \leq 2\pi, -\frac{\pi}{2} \leq \psi \leq
  \frac{\pi}{2}$ является сферой с центром в начале координат и радиусом $R$ у
  которой весь меридиан $\varphi = 0$ состоит из кратных точек.
\end{example}

\section{Касательная плоскость и нормаль к поверхности в $\mathbb{R}^3$}

Пусть $S = \{\pvec r(u, v) : (u, v) \in \overline{\mathbb{D}}\}$ --- непрерывно
диффиренцируемая поверхность, $(u_0, v_0) \in \mathbb{D}, \pvec r_u'$ ---
производная вектор-функции $\pvec r = \pvec r(u, v_0)$, то есть $\pvec r_u'$ ---
касательный вектор к кривой $\pvec r = \pvec r(u, v_0)$, называемой координатной
линией, $\pvec r_v'$ --- касательный вектор к координатной линии $\pvec r = \pvec
r(u_0, v)$.

\begin{definition}
  Точка $M_0 = \pvec r(u_0, v_0)$ поверхности $S$ называется неособой, если в
  ней векторы $\pvec r_u', \pvec r_v'$ не колинеарны: $[\pvec r_u', \pvec r_v']
  \not = 0$ в противном случае точка $M_0$ называется особой.
\end{definition}

\begin{definition}
  Плоскость, проходящая через неособую точку $M_0 = \pvec r(u_0, v_0)$
  поверхности $S$, параллельно векторам $\pvec r_u', \pvec r_v'$ называется
  касательной плоскостью к поверхности $S$ в этой точке. Если $\pvec r_0 = \pvec
  r(u_0, v_0), \pvec r$ --- радиус вектор произвольной точки на касательной
  плоскости, то ее уравнение в векторной записи имеет вид:
  \begin{gather}
    < \pvec r - \pvec r_0, \pvec r_u', \pvec r_v' > = 0
  \end{gather}
\end{definition}

\begin{definition}
  Если $M_0 = \pvec r(u_0, v_0)$ --- неособая точка, то вектор
  \begin{gather}
    \pvec\nu = \frac{[\pvec r_u', \pvec r_v']}{\left|[\pvec r_u', \pvec
    r_v']\right|}
    \label{def723:eq1}
  \end{gather}
  а так же ему противоположный назовем единичной нормалью к поверности в точке
  $M_0$.
\end{definition}

\begin{definition}
  Непрерывно дифиренцируемая поверхность без особых точек называется гладкой
  поверхностью. Объединение конечного числа гладких поверхностей назовем
  кусочно гладкой поверхностью, она может состоять из парочки кусков.
\end{definition}

\section{Площадь поверхности}
Пусть
\begin{gather}
  S = \{\pvec r(u, v) : (u, v) \in \overline{\mathbb{D}}\}
  \label{ch73:eq1}
\end{gather}
непрерывно диффиренцируемая поверхность, где $\overline{\mathbb{D}}$ ---
квадрируемая замкнутая область. Пусть $I$ --- промежуток, содержащий множество
$\overline{\mathbb{D}}$. $T$ --- некоторое разбиение промежутка $I$.
Прономеруем каким-либо образом те промежутки разбиения, которые содержатся в
$\overline{\mathbb{D}}$ и обозначим их $I_j, \ j = 1, \dots, k$. \\ Возьмем
какой-либо промежуток $I_j$.
Пусть $I_j = [u, u + h] \times [v, v + t], t > 0,
h > 0$ и где для краткости записи пропущен индекс ${}_j$ у переменных: $u, v, h,
t$. Тогда
\begin{gather*}
  \pvec r(u + h, v) - \pvec r(u, v) = \pvec r_u' \cdot h +
  \bar{\bar{o}}\left(h\right), \ h \to 0 \\
  \pvec r(u, v + t) - \pvec r(u, v) = \pvec r_v' \cdot t +
  \bar{\bar{o}}\left(t\right), \ t \to 0
\end{gather*}
При определении площади поверхности образы промежутков $I_j$ будем заменять
прямолинейными параллелограмами, построенными на векторах $\pvec r_u' \cdot h,
\ \pvec r_v' \cdot t$. Обозначим площадь этого параллелограма
$\triangle\delta_i$, получаем:
\begin{gather*}
  \triangle \delta_i = |[\pvec r_u' \cdot h, \pvec r_v' \cdot t]| = |[\pvec
  r_u', \pvec r_v']| ht = |[\pvec r_u', \pvec r_v']|_{\mu_j}|I_j|, \ \mu_j =
  (u_j, v_j)
\end{gather*}
Наряду с поверхностью $S$ рассмотрим чешуйчатую поверхность, составленную из
всех параллелограмов, построенных для каждого прямоугольника $I_j, \ (j = 1,
\dots, k)$ на соответуствующих ему векторах $\pvec r_u', \pvec r_v'$. Ее
площадь равна:
\begin{gather*}
  \sum\limits_{j = 1}^{k} |[\pvec r_u', \pvec r_v']|_{\mu_j}|I_j|
\end{gather*}
которую можно считать приближенным значением поверхности $S$, при чем все более
точным при $\lambda(T) \to 0$. Таким образом мы принимаем определение
\eqref{def731}.

\begin{definition}
  \label{def731}
  Площадью $\mu_2(S)$ поверхности $S$ называется величина:
  \begin{gather}
    \mu_2(S) := \iint\limits_{\mathbb{D}} |[\pvec r_u', \pvec r_v']| du dv
    \label{def731:eq1}
  \end{gather}
\end{definition}

\begin{lemma}[тождество Лагранжа]
  Для любых двух векторов:
  \begin{gather*}
    |[\pvec a, \pvec b]|^2 = \det
    \begin{pmatrix}
      (\pvec a, \pvec a) & (\pvec a, \pvec b) \\
      (\pvec b, \pvec a) & (\pvec b, \pvec b)
    \end{pmatrix} =
    |\pvec a|^2 |\pvec b|^2 - (\pvec a, \pvec b)^2
  \end{gather*}
  называется тождеством Лагранжа.
\end{lemma}

\begin{proof}
  Для доказательства достаточно: $|[\pvec a, \pvec b]| = |\pvec a||\pvec b|
  \sin \widehat{\pvec a \pvec b}, \ (\pvec a, \pvec b) = |\pvec a||\pvec b|
  \cos \widehat{\pvec a \pvec b}$. Детерминант в лемме называется детерминантом
  Грамма векторов $\pvec a, \pvec b$.
\end{proof}

Введем обозначение:
\begin{align}
  E &= g_{11} = |(\pvec r_u', \pvec r_u')| \nonumber \\
  F &= g_{12} = g_{21} = |(\pvec r_u', \pvec r_v')|
  \label{ch73:eq2} \\
  G &= g_{22} = |(\pvec r_v', \pvec r_v')| \nonumber
\end{align}
Из леммы $\pvec a = \pvec r_u', \ \pvec b = \pvec r_v'$. Из леммы $|[\pvec
r_u', \pvec r_v']| = EGF^2 = \det (g_{ij})$, поэтому формула \eqref{def731:eq1}
примет вид:
\begin{gather}
  \mu_2(S) = \iint\limits_{\mathbb{D}} \sqrt{EG - F^2} du dv \equiv \iint
  \sqrt{\det(g_{ij})} du dv
  \label{ch73:eq3}
\end{gather}
в частности, если $S$ --- график функции $z = f(x, y), (x, y) \in
\overline{\mathbb{D}}$, то $u = x, \ v = y, \\ \pvec r = \{x, y, f(x, y)\}$ и
следовательно $\pvec r_u' = \{1, 0, f_x'\}, \ \pvec r_v' = \{-1, 1, f_y'\}$.
Тогда получим:

\begin{gather*}
  \det (g_{ij}) = \det
  \begin{pmatrix}
    1 + f_x'^2 & f_x' \cdot f_y' \\
    f_x' \cdot f_y' & 1 + f_y'^2
  \end{pmatrix} =
  1 + f_x'^2 + f_y'^2 \\
  \mu_2(S) = \iint\limits_{\mathbb{D}} \sqrt{1 + f_x'^2 + f_y'^2} dx dy
\end{gather*}

\begin{remark}
  Использую формулу замены переменного в двойном интеграле не трудно доказать,
  что определение \eqref{def731} площади поверхности не зависит от выбора ее
  представления.
\end{remark}

\section{Ориентация поверхности в $\mathbb{R}^3$}
Рассмотрим $\mathbb{R}^3$, то есть $Oxyz$. Пусть $S$ --- гладкая поверхность
\begin{gather}
  \pvec r = \pvec r(u, v): (u, v) \in \overline{\mathbb{D}}
  \label{ch74:eq1}
\end{gather}
ее векторное представление.
\begin{gather}
  \pvec \nu = \frac{\pvec n}{|\pvec n|}, \pvec n = [\pvec r_u', \pvec r_v']
  \label{ch74:eq2}
\end{gather}
$\pvec \nu$ --- ее единичная нормаль. Поскольку представление $(u, v)$
непрерывно диффиренцируемо, то вектор $\pvec \nu$ (так же как и вектор $-\pvec
\nu$) является непрерывной функцией на $\overline{\mathbb{D}}$.

\begin{definition}
  Всякая непрерывная на $\overline{\mathbb{D}}$ единичная нормаль $\pvec \nu =
  \pvec \nu(u, v)$ гладкой поверхности $S$ называется ориентацией поверхности
  $S$, если $\pvec \nu$ является так же однозначной непрерывной функцией
  переменной точки $\mu(x, y, z) = \pvec r(u, v)$ на самой поверхности $S$. В
  этом случае поверхность $S$ называется ориентированной (или двусторонней), в
  противном случае неориентированной (односторонней). Поверхность у которой
  фиксированна одна из ориентаций, называется ориентированной стороной
  поверхности.
\end{definition}
Очевидно, что поверхность может иметь только две ориентации, который называют
противоположными, одна положительная, другая отрицательная. $\mu(x, y, z) =
\pvec r(u, v)$. Примером ориентации поверхности является всякая простая гладкая
поверхность. Ориентированной является и сфера(хотя сфера и не является простой
поверхностью). Примером неориентированной поверхности является лента Мебиуса,
непрерывная поверхность в определении нарушается. \\

Пусть гладкая ориентированная поверхность $S$ с представлением $\pvec r = \pvec
r(u, v), \ (u, v) \in \overline{\mathbb{D}}$, ориентирована нормалью $\pvec \nu
= \frac{\pvec n}{|\pvec n|}$, где $\pvec n = [\pvec r_u', \pvec r_v']$. И пусть
граница $\partial \mathbb{D}$ в области $\mathbb{D}$ является непрерывной
кривой, ориентированной положительно:
\begin{gather*}
  \partial \mathbb{D} = \{u(t), v(t) : a \leq t \leq b\}
\end{gather*}
Очевидно, что эта ориентация порождает определенную ориентацию края $\partial
S = \{\pvec r = \pvec r(u(t), v(t)) : a \leq t \leq b\}$.

\begin{definition}
  Указанная выше ориентация края $\partial S$ поверхности $S$ называется
  согласованной с ориентацией $\pvec \nu$ поверхности $S$. Для простой
  поверхности, в частности для графика $z = f(x, y)$ вектор нормали $\pvec
  \nu (-\pvec \nu)$ согласован с положительной (отрицательной) ориентацие
  кривой $\partial S$ по правилу штопора.
\end{definition}
Ориентация контура $\partial S$ соответствует направлению вращения ручки
штопора, а направлению нормали соответствует движение штопора.

\begin{definition}
  \label{def743}
  Кусочно-гладкая поверхность $S = S_1 \cap S_2 \cap \dots \cap S_k$ называется
  ориентированной, если ее можно представить как результат такой склейки
  гладких поверхностей $S_i, i = 1, \dots, k$, при которой общие части краев
  $\partial S_i$ поверхности $S_i$ принадлежат не более, чем двум этим
  поверхностям и проходят в противоположных направлениях, при ориентации краев
  $\partial S_i$, согласованных по правилу штопора с ориентациями указанных
  двух поверхностей. Объединение краев $\partial S_i$, принадлежащих одному
  такому краю, называется краем поверхности $S$.
\end{definition}

Примерами кусочно-гладких ориентированных(двусторонних) поверхностей являются
поверхности параллелепипедов, цилиндров. \\
Можно показать, что определение \eqref{def743} в важном частном случае
совпадает со следующим определением.

\begin{definition}
  Если кусочно-гладкая поверхность $S$ является границей области $\mathbb{G}
  \subset \mathbb{R}^3$, то еденичная нормаль $\pvec \nu$ к этой поверхности
  (там где она существует), направленная внутрь области называется внутренней
  нормалью (относительно области $\mathbb{G}$), а противоположная нормаль
  $-\pvec \nu$ --- внешней нормалью. Эти нормали называются ориентациями
  границы $\partial \mathbb{G}$ области $\mathbb{G}$.
\end{definition}

\section{Определение поверхностного интеграла первого рода}
Пусть $S = \{\pvec r(u,v) : (u, v) \in \overline{\mathbb{D}}\}$ --- гладкая
поверхность, $\mathbb{D}$ --- квадрируемая область. $F$ --- функция, заданная
на поверхности $S$, то есть $F = F(\pvec r(u, v)) = F(x(u, v), y(u, v), z(u,
v)), (u, v) \in \overline{\mathbb{D}}$.

\begin{definition}
  Интеграл первого рода $\iint\limits_S \mathcal{F} dS$ по поверхности $S$ называется
  интеграл:
  \begin{gather}
    \iint\limits_S \mathcal{F}(x, y, z) dS := \iint\limits_D \mathcal{F}(\pvec r(u, v)) \sqrt{EG -
    F^2} du dv
    \label{def751:eq1}
  \end{gather}
  Если функция $\mathcal{F}$ --- непрерывная на поверхности $S$, то интеграл
  \eqref{def751:eq1} безусловно существует.
\end{definition}

\begin{example}
  Функция $\mathcal{F} \equiv 1$ на поверхности $S$, тогда получаем:
  \begin{gather*}
    \iint\limits_S dS = \iint\limits_{\mathbb{D}} \sqrt{EG - F^2} du dv =
    \mu_2(S)
  \end{gather*}
\end{example}

\begin{example}
  Если поверхность $S$ задана явно, $z = f(x, y), \ (x, y) \in
  \overline{\mathbb{D}}$, то:
  \begin{gather*}
    \iint\limits_S \mathcal{F}(x, y, z) dS = \iint\limits_{\mathbb{D}}
    \mathcal{F}(x, y, f(x, y)) \sqrt{1 + f_x'^2 + f_y'^2} dx dy
  \end{gather*}
\end{example}

\begin{definition}
  Поверхностный интеграл первого рода по кусочно-гладкой поверхности
  определяется как сумма интегралов по ее гладким частям.
\end{definition}

\section{Поверхностный интеграл второго рода}
Пусть $\pvec \nu = \{\cos \alpha, \cos \beta, \cos \gamma\}$ --- непрерывная
еденичная нормаль на ориентированной гладкой поверхности $S$, представление
которой задано в квадрируемой области. Ориентированную с помощью этой нормали
поверхность обозначим через $S^+$. Пусть $\pvec a = \pvec a(x, y, z) = \{P, Q,
R\}$ --- векторная функция заданная на поверхности $S$ (так что $(P, Q, R)$ ---
числовые функции на поверхности $S$).

\begin{definition}
  Поверхностным интегралом второго рода по ориентированной поверхности $S^+$
  называется интеграл: $\iint\limits_{S^+} \pvec a d \pvec S := \iint\limits_S
  (\pvec a, \pvec \nu) dS$. Для интегралов $\iint\limits_S \pvec a d \pvec S$
  используется также обозначение:
  \begin{gather}
    \iint\limits_{S^+} P dy dz + Q dx dz + R dx dy
    \label{def761:int1}
  \end{gather}
  таким образом получим:
  \begin{gather}
    \iint\limits_{S^+} P dy dz + Q dx dz + R dx dy = \iint\limits_S (P \cos
    \alpha + Q \cos \beta + R \cos \gamma) dS
    \label{def761:eq1}
  \end{gather}
\end{definition}

Для случая, когда поочередно две функции из $P, Q, R$ тождественно равны нулю,
очевидны интегралы:
\begin{gather}
  \iint\limits_{S^+} P dy dz := \iint\limits_S P \cos \alpha dS \nonumber \\
  \iint\limits_{S^+} Q dx dz := \iint\limits_S Q \cos \beta dS
  \label{ch76:eqs1} \\
  \iint\limits_{S^+} R dx dy := \iint\limits_S R \cos \gamma dS \nonumber
\end{gather}
Выражение \eqref{def761:int1} теперь можно рассматривать как сумму трех только
что определенных интегралов. \\

Следующие утверждения очевидны:
\begin{enumerate}
  \item $\pvec a$ непрерывен на поверхности $S$, то $\iint\limits_{S^+} \pvec a
    d \pvec S$ существует.
  \item Если ориентированную с помощью вектора $-\pvec \nu$ поверхность $S$
    обозначить через $S^-$, то:
    \begin{gather*}
      \iint\limits_{S^-} \pvec a d \pvec S = - \iint\limits_{S^+} \pvec a d
      \pvec S
    \end{gather*}
\end{enumerate}

\begin{theorem}
  Пусть $S = \{\pvec r(u, v), \ (u, v) \in \overline{\mathbb{D}}\}$ --- гладкая
  поверхность, $S^+$ --- поверхность $S$, ориентированная с помощью вектора
  $\pvec \nu = \frac{\pvec n}{|\pvec n|}$, где $\pvec n = [\pvec r_u', \pvec
  r_v']$, тогда поверхностный интеграл второго рода:
  \begin{gather}
    \iint\limits_{S^+} \pvec a d \pvec S = \iint\limits_{\mathbb{D}} <\pvec a,
    \pvec r_u', \pvec r_v'> du dv = \iint\limits_{\mathbb{D}}
    \begin{vmatrix}
      P & Q & R \\
      x_u' & y_u' & z_u' \\
      x_v' & y_v' & z_v'
    \end{vmatrix}
    du dv
    \label{th761:eq1}
  \end{gather}
  где $P = P(x(u, v), y(u, v), z(u, v)), \ Q = Q(x(u, v), y(u, v), z(u,v)), \ R =
  R(x(u, v), y(u, v), z(u,v)), \\ \pvec r_u' = \{x_u', y_u', z_u'\}, \ \pvec r_v' =
  \{x_v', y_v', z_v'\}$. В частности, полагая поочередное равенство двух
  функций и $P, Q, R$ нулю, найдем интегралы \eqref{ch76:eqs1}.
\end{theorem}

\begin{proof}
  \begin{gather*}
    \iint\limits_{S^+} \pvec a d \pvec S := \iint\limits_S (\pvec a, \pvec \nu)
    dS = \iint\limits_{\mathbb{D}} \left(\pvec a, \frac{\pvec n}{|\pvec n|}
    \right) |\pvec n| du dv = \iint\limits_{\mathbb{D}} <\pvec a, \pvec r_u',
    \pvec r_v'> du dv
  \end{gather*}
\end{proof}

\begin{example}
  \label{ex761}
  Поверхность $S$ имеет явное представление: $z = f(x, y), \ (x, y) \in
  \overline{\mathbb{D}}$. Тогда $x = u, y = v, z = f(u, v), \ (u, v) \in
  \overline{\mathbb{D}}$ --- ее параметрическое представление:
  \begin{gather}
    \begin{vmatrix}
      P & Q & R \\
      x_u' & y_u' & z_u' \\
      x_v' & y_v' & z_v'
    \end{vmatrix} =
    \begin{vmatrix}
      \pvec i & \pvec j & \pvec k \\
      1 & 0 & f_u' \\
      0 & 1 & f_v'
    \end{vmatrix} =
    \{-f_u', -f_v', 1\}
    \label{ex761:eq1}
  \end{gather}
  Поэтому, если $\pvec a = \{0, 0, R\}$, то $<\pvec a, \pvec r_u', \pvec r_v'>
  = (\pvec a, \pvec n) = R$. Отсюда из формулы \eqref{th761:eq1} получаем:
  \begin{gather}
    \iint\limits_{S^+} R dx dy := \iint\limits_{S^+} \pvec a d \pvec S =
    \iint\limits_{\mathbb{D}} R(x, y, f(x, y)) dx dy
    \label{ex761:eq2}
  \end{gather}
\end{example}

\begin{remark}
  Поскольу в примере \eqref{ex761} проекция вектора $\pvec n$ на вектор $\pvec
  k$ равна еденице, то $\cos \gamma = \text{Пр}_{\pvec k} \pvec \nu > 0$.
  $\pvec \nu$ образует угол с осью $Oz$, то есть направленно вверх от
  поверхности $S$, поэтому поверхность $S^+$ называют верхней стороной $S$, а
  противположно ориентированную поверхность $S^-$ --- нижней стороной.
\end{remark}

\begin{example}
  Пусть $S_0$ --- цилиндрическая поверхность, направляющей которой является
  некоторая гладкая кривая, лежащая в плоскости $Oxy$, а образующая
  параллельная оси $Oz$. $S_0^+$ --- поверхность $S_0$, ориентированная
  непрерывной нормалью $\pvec \nu$, тогда $\cos \gamma = \text{Пр}_{Oz}\pvec
  \nu = 0$ и
  \begin{gather}
    \iint\limits_{S_0} R dx dy = \iint\limits_{S_0} R\cos \gamma dS = 0
    \label{ex762:eq1}
  \end{gather}
\end{example}

\begin{definition}
  \label{def762}
  Поверхностный интеграл второго рода по кусочно-гладкой ориентированной
  поверхности $S$ определим как сумму интегралов второго рода по гладким
  частям, составим эту поверхность, при условии, что ориентация каждой из этих
  частей совпадает с выбранной ориентацие поверхности.
\end{definition}

