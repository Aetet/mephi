\section{Определение интеграла Римана на $n$-ом промежутке}
Пусть $\mathbb{R}^n$ --- $n$-мерное арифметическое евклидово пространство(ЕП).
\\ ($\mathbb{R}^2$ отождествляется $oxy$, $\mathbb{R}^3$ с $oxyz$).\\
$(x_1, \dots, x_n) \in \mathbb{R}^n$ далее обозначается через $x$.

\begin{definition}
  Множество $I = \{x \in \mathbb{R}^n : a_i \leq x \leq b_i, \ i = 1, \dots, n\}$
  называется промежутком или координатным параллелепипедом. По аналогии с
  одномерным случаем записывают:
  \begin{gather*}
    a = (a_1, \dots, a_n), \ b = (b_1, \dots, b_n)
  \end{gather*}
\end{definition}

\begin{definition}
  Число $|I| := \prod\limits_{i=1}^{n} (b_i - a_i)$ называют объемом, либо мерой
  промежутка. \\
  Объем(меру) обозначают символами: $v(I), \mu(I)$.
  \begin{align*}
    \text{При} \ n &= 1, \mu(I) \text{ --- длина отрезка} \\
    \text{При} \ n &= 2, \mu(I) \text{ --- площадь прямоугольника} \\
    \text{При} \ n &= 3, \mu(I) \text{ --- объем прямоугольного параллелепипеда}
  \end{align*}
  Разобьем каждый из координатных отрезков $[a_i, b_i], \ i = 1, \dots, n$ на
  конечное число более мелких отрезков. Эти разбиения индуцируют разбиение
  промежутка $I$ на более мелкие промежутки, получающиеся прямым произведением
  промежутков $a_i, b_i$.
\end{definition}

\begin{definition}[Декартово произведение]
  $X \times Y := \{(x, y) : x \in X, y \in Y \}$
\end{definition}

\begin{definition}
  Описанное представление промежутка $I$ в виде объеденения промежутков $i, j$
  из более мелких промежутков $i, j$ назовем разбиением и обозначим $T(k =
  k_T)$.
\end{definition}

\begin{definition}
  $\lambda(T) = \max\{d(I_1), \dots, d(I_k)\}$.
  $d(I_j)$ называется диаметром разбиения~$T$. Пусть $f(x) = f(x_1, \dots,
  x_n)$ --- функция, определенная на промежутке $I, \\ T = \{I_1, \dots, I_k\}$
  --- разбиение промежутка $I$. \\
  $\xi = (\xi^1, \dots, \xi^k)$ --- некоторый набор точек $\xi^j$, таких что
  $\xi^j \in I_j$.
\end{definition}

\begin{definition}
  Сумма $\sigma(f, T, \xi) := \sum\limits_{j = 1}^k f(\xi^j)|I_j|$ называется
  интегральной суммой Римана.
\end{definition}

\begin{definition}
  Если существует конечный предел $\mathcal{J}$, $\sigma(f, T, \xi), \
  \lambda(T) \to 0$, то его называют интегралом Римана от функции $f$ на
  промежутке $I$:
  \begin{gather}
    \mathcal{J} = \int\limits_I f(x) dx := \lim\limits_{\lambda(T) \to 0}
    \sigma(f, T, \xi)
    \label{def517:eq1}
  \end{gather}
  Функцию $f$ называют интегрируемой на промежутке $I$. Множество таких функций
  обозначим $\mathcal{R}(I)$. Уточним, что равенство \eqref{def517:eq1}
  означает:
  $\forall \varepsilon > 0 \ \exists \delta > 0 : \forall T = \{I_1, \dots,
  I_k\}, \lambda(T) < \delta$ и при любом выборе точек $\xi^j \in I_j, j = 1,
  \dots, k$ выполняется неравенство:
  \begin{gather*}
    \left|\mathcal{J} - \sum\limits_{j = 1}^{k} f(\xi^j) |I_j|\right| <
    \varepsilon
  \end{gather*}
  Равносильные отображения интеграла таковы:
  \begin{gather*}
    \int\limits_I f(x_1, \dots, x_n) \ dx_1 \dots dx_n, \
    \underset{n}{\underbrace{\int \dots \int}} f(x_1, \dots, x_n) \ dx_1 \dots
    dx_n, \ n \in \mathbb{N}
  \end{gather*}
\end{definition}

Мы видим, что данное определение повторяет определение интеграла Римана, и при
$m = 1$ совпадает. Схожесть опеределений позволяет найти сходные методы о
решении вопроса условий существования.

\section{Условие существования кратного интеграла}

\begin{theorem}
  Если $f \in \mathcal{R}(I)$, то $f$ ограниченна на $I$.
  Пусть функция $f$ определена на $I, \ I \in \mathbb{R}^n$. $T = \{I_j\}$ ---
  разбиение промежутка $I$.
  \begin{gather*}
    m_j = \inf\limits_{x \in I_j} f(x), \ M_j = \sup\limits_{x \in I_j} f(x)
  \end{gather*}
\end{theorem}

\begin{definition}
  Величины $s(f, T) = \sum\limits_{j} m_j |I_j|, \ S(f,T) = \sum\limits_{j} M_j
  |I_j|$ называются соотвтетственно нижней и верхней суммой Дарбу на промежутке
  отвечающему разбиению $T$ этого промежутка.
\end{definition}

\begin{comment}
  Совершенно аналогично доказывается при $m = 1$ теорема \eqref{th522}.
\end{comment}

\begin{theorem}
  \label{th522}
  Для того, чтобы ограниченная на промежутке функция $f$ была интегрируема на
  $I \Longleftrightarrow$
  \begin{gather}
    \lim\limits_{\lambda(T) \to 0} (S(f, T) - s(f, T)) = 0
    \label{th522:eq1}
  \end{gather}
\end{theorem}

\begin{remark}[критерий интегрируемости Римана]
  Если обозначить колебания $M_j - m_j, \ I_j$ через $\Omega(f, I_j)$, то
  \eqref{th522:eq1} можно записать в виде:
  \begin{gather*}
    \lim\limits_{\lambda(T) \to 0} \sum\limits_{j} \Omega(f, I_j) |I_j| = 0
  \end{gather*}
\end{remark}

\begin{definition}
  Говорят, что множество $\mathbb{E} \ n$-мерного пространства имеет меру $0$ в
  смысле Жордана или имеет нулевой $n$-мерный объем, если для $\forall
  \varepsilon > 0$ существует покрытие множества $\mathbb{E}$ конечной системой
  $\{I_j\} \ n$-мерных промежутков. $\sum\limits_{j} |I_j|$ объемов которых
  меньше~$\varepsilon$. В этом случае пишем: $\mu(\mathbb{E}) = 0$
\end{definition}

\begin{theorem}
  \label{th523}
  Пусть функция $f$ ограничена на $n$-мерном промежутке $I \subset
  \mathbb{R}^n$ и $\mathbb{E}_f$ --- множество ее точек разрыва. Тогда, если
  $\mu(\mathbb{E}_f) = 0$, то $f \in \mathcal{R}(I)$ (интегрируема).
\end{theorem}

\begin{comment}
  Эта теорема при $n = 1$ обычно устанавливается в разделе ``определенный
  интеграл''. В общем случае доказывается аналогично.
\end{comment}

\section{Кратный интеграл по множеству}
Пусть функция $f$ определена на $\mathbb{E} \subset \mathbb{R}^n$, условимся
символом $f_{\mathbb{E}}(x)$ обозначать функцию $= 0$ вне $\mathbb{E}$.
\begin{definition}
  \label{def531}
  Интеграл функции по множеству $\mathbb{E}$ определяется соотношением:
  \begin{gather}
    \int\limits_{\mathbb{E}} f(x) dx := \int\limits_I f_{\mathbb{E}}(x) dx
    \label{def531:eq1}
  \end{gather}
  где $I$ --- наименьший промежуток, содержащий $\mathbb{E}$. \\
  Если стоящий справа интеграл существует, то он называется интегрируемым по
  Риману на~$\mathbb{E}$.
  Совокупность всех интегрируемых на множестве $\mathbb{E}$ функций обозначим
  $\mathcal{R}(\mathbb{E})$.
\end{definition}

\begin{definition}
  Множество $\mathbb{E} \subset \mathbb{R}^n$ назовем допустимым, если оно
  ограничено в $\mathbb{R}^n$ и его граница $\partial \mathbb{E}$ есть
  множество меры нуль (в смысле Жордана).
\end{definition}

\begin{example}
  Куб, тетраэдер, шар и т. д. являются допустимыми множествами.
\end{example}

\begin{comment}
  Граница $\partial \mathbb{E}$ множества $\mathbb{E}$ состоит из точек, в
  любой окрестности которых имеются как точки из $\mathbb{E}$, так и точки из
  дополнения $\mathbb{E}$ до $\mathbb{R}^n$.
\end{comment}

\begin{theorem}
  Пусть $\mathbb{E}$ --- допустимое множество в $\mathbb{R}^n$. $f$ ---
  функция, определенная на $\mathbb{E}$ и пусть множество точек разрыва
  $\mathbb{E}_f$ множества $\mathbb{E}$ имеют нулевую меру, тогда функция $f
  \in \mathcal{R}(\mathbb{E})$ (интегрируема).
\end{theorem}

\begin{proof}
  Функция $f_{\mathbb{E}}$ по сравнению с функцией $f$ может иметь
  дополнительные разрывы на $\partial \mathbb{E}$, которые, по условию, имеет
  меру нуль. Поэтому множество точек разрыва функции $f_{\mathbb{E}}$ так же
  имеет нулевую меру. \\
  Отсюда из определения \eqref{def531} и теоремы \eqref{th523} следует, что
  функция интегрируема.
\end{proof}

\section{Мера(объем) множества}
\begin{definition}
  Мерой Жордана(или объемом) ограниченного множества $\mathbb{E}$ назовем
  величину $\mu(\mathbb{E}) := \int\limits_{\mathbb{E}} dx$, если указанный
  интеграл(Римана) существует. В последнем случая множество $\mathbb{E}$
  называют измеримым в смысле Жордана.
\end{definition}

\begin{theorem}
  Допустимое множество измеримо в смысле Жордана.
\end{theorem}

\begin{proof}
  Рассмотрим характеристическую функцию:
  \begin{gather*}
    \aleph_{\mathbb{E}}(x) =
    \begin{cases}
      1, \ x \in \mathbb{E} \\
      0, \ x \not \in \mathbb{E}
    \end{cases}
  \end{gather*}
  \begin{comment}
    $\aleph$ --- Алеф.
  \end{comment}
  Очевидно, функция $\aleph_{\mathbb{E}}(x)$ имеют разрывы в граничных и только
  в граничных точках множества $\mathbb{E}$. \\
  По определению \eqref{def531}:
  \begin{gather}
    \int\limits_{\mathbb{E}} 1 \cdot dx = \int\limits_I \aleph_{\mathbb{E}}(x)
    dx
    \label{th541:eq1}
  \end{gather}
  где $I$ --- наименьший промежуток, содержащий множество $\mathbb{E}$. \\
  А так как множество точек разрыва $\aleph_{\mathbb{E}}(x)$ совпадает с
  границей $\partial \mathbb{E}$ и $\mu(\partial \mathbb{E}) = 0$, то по
  теореме \eqref{th523} интеграл \eqref{th541:eq1} существует. \\
  $\mu(\mathbb{E})$ носит комплексный смысл, если $\mathbb{E}$ --- измеримое
  множество, и
  \begin{gather*}
    \int\limits_I \aleph_{\mathbb{E}}(x) dx = \lim\limits_{\lambda(T) \to 0}
    s(\aleph_{\mathbb{E}}, T) = \lim\limits_{\lambda(T) \to 0}
    S(\aleph_{\mathbb{E}}, T), \ (I > \mathbb{E})
  \end{gather*}
  то
  \begin{gather*}
    s(\aleph_{\mathbb{E}}, T) = \sum\limits_{j} m_j |I_j|, \
    S(\aleph_{\mathbb{E}}, T) = \sum\limits_{j} M_j |I_j|
  \end{gather*}
  где $s, S$ --- нижняя и верхняя суммы Дарбу. Но в силу определения функции
  $\aleph_{\mathbb{E}}: s(\aleph_{\mathbb{E}}, T)$ равна сумме объемов
  промежутков $I_j$, лежащих в множестве $\mathbb{E}$ (это объем вписанного в
  $\mathbb{E}$ многогранника), а $S(\aleph_{\mathbb{E}}, T)$ равна сумме
  объемов тех промежутков $I_j$, которые имеют общие точки с множеством
  $\mathbb{E}$ (Объем описанного многогранника).
\end{proof}

\begin{approval}
  Мера $\mu(\mathbb{E})$ есть общий предел, при $\lambda(T) \to 0$ объемов,
  вписанных в $\mathbb{E}$ и описанных около $\mathbb{E}$ многогранников.
\end{approval}

\begin{remark}
  Можно показать, что измеримы по Жордану только измеримые множества
  $\Longleftrightarrow$ множество $\mathbb{E}$ является измеримым по Жордану
  $\Longleftrightarrow$ его границы имеют меру нуль в смысле Жордана.
\end{remark}

\begin{remark}
  При $n = 2$ понятие измеримого по Жордану множества совпадает с понятием
  квадрируемой плоской фигуры и меры Жордана с ее площадью.
\end{remark}

\section{Свойства кратных интегралов}
На кратные интегралы по ограниченной функции переносятся все свойства
интегралов по отрезку. \\
Доказатаельства аналогичны одномерному случаю:
\begin{enumerate}
  \item \underline{Линейность} интеграла по множеству. \\
    $f_1, \dots, f_n$ интегрируема на множестве $\mathbb{E}, \ \lambda_1, \dots, \lambda_n$
    \\
    $\lambda_1 f_1, \dots, \lambda_n f_n$ так же интегрируема на $\mathbb{E}$ и
    \begin{gather*}
      \int\limits_{\mathbb{E}} \left(\lambda_1 f_1(x) + \dots + \lambda_n
      f_n(x) \right) dx = \lambda_1 \int\limits_{\mathbb{E}} f_1(x) dx + \dots
      \lambda_n \int\limits_\mathbb{E} f_n(x) dx
    \end{gather*}
  \item \underline{Аддитивность} интеграла по множеству. \\
    Если $\mathbb{E}_1, \mathbb{E}_2$ --- допустимые множества в $\mathbb{R}^n$
    и $\mu(\mathbb{E}_1 \cap \mathbb{E}_2) = 0 \ (\mathbb{E}_1 \cap \mathbb{E}_2
    \not = 0$ в частности), а $f$ --- функция, определенная на $\mathbb{E}_1
    \cup \mathbb{E}_2$, то при условии существования интегралов имеет место
    равенство:
    \begin{gather*}
      \int\limits_{\mathbb{E}_1 \cup \mathbb{E}_2} f(x) dx =
      \int\limits_{\mathbb{E}_1} f(x) dx + \int\limits_{\mathbb{E}_2} f(x) dx
    \end{gather*}
  \item \underline{Общая оценка}. \\
    Если $f \in \mathcal{R}(\mathbb{E})$, то $|f| \in \mathcal{R}(\mathbb{E})$ и имеет место
    неравенство:
    \begin{gather*}
      \left| \int\limits_{\mathbb{E}} f(x) dx \right| \leq
      \int\limits_{\mathbb{E}} |f(x)| dx
    \end{gather*}
  \item \underline{Интегрирование неравенств}.
    Если функции $f, g \in \mathcal{R}(\mathbb{E})$ и $f(x) \leq g(x), \ x \in
    \mathbb{E}$, то:
    \begin{gather*}
      \int\limits_\mathbb{E} f(x) dx \leq \int\limits_\mathbb{E} g(x) dx
    \end{gather*}
  \item \underline{Следствие из 4}. \\
    Если $f$ интегрируема и $m \leq f(x) \leq M, x \in \mathbb{E}$
    \begin{gather*}
      m\mu(\mathbb{E}) \leq \int\limits_\mathbb{E} f(x)dx \leq M\mu(\mathbb{E})
    \end{gather*}
  \item \underline{Теорема о среднем}. \\
    Если в дополнении условия 5. множество $\mathbb{E}$ линейно связано, а $f$
    --- непрерывна на $\mathbb{E}$, то существует $\xi \in \mathbb{E}$, такое
    что:
    \begin{gather*}
      \int f(x) dx = f(\xi) \mu(\mathbb{E})
    \end{gather*}
\end{enumerate}

\section{Сведение кратного инетграла к повторному}
\begin{theorem}
  $X \times Y \subset \mathbb{R}^{m+n}$ является прямым произведением
  промежутков: $X \subset \mathbb{R}^m$ и $Y \subset \mathbb{R}^n$. \\
  Если для $f(x, y)$ определенной на $X \times Y, (x \in X, y \in Y)$
  существует интеграл:
  \begin{gather}
    \int\limits_{X \times Y} f(x, y) dx dy
    \label{th561:ex1}
  \end{gather}
  и для любого $x \in X$ существует:
  \begin{gather}
    \mathcal{J}(x) = \int\limits_Y f(x, y) dy
    \label{th561:eq1}
  \end{gather}
  то существует так же и повторный интеграл:
  \begin{gather}
    \int\limits_X dx \int\limits_Y f(x, y) dy
    \label{th561:ex2}
  \end{gather}
  и выполняется равенство:
  \begin{gather}
    \int\limits_{X \times Y} f(x, y) dx dy = \int\limits_X dx \int\limits_Y
    f(x, y) dy
    \label{th561:eq2}
  \end{gather}
\end{theorem}

\begin{proof}
  Любое разбиение $T$ промежутка $X \times Y$ индуцируется собственными
  разбиениями $T_x, T_y$. При этом, каждый промежуток разбиения $T$ есть прямое
  произведения $X_i \times Y_j, \ X_i, Y_j$ --- разбиения $T_x, T_y$. \\
  Очевидно, $|X_i \times Y_j| = |X_i| \cdot |Y_j|$. \\
  Положим $m_{ij} = \inf\limits_{x \in X_i, \ y \in Y_j} f(x, y), \ M_{ij} =
  \sup\limits_{x \in X_i, \ y \in Y_j} f(x, y)$, так что выполняется следующее
  неравенство:
  \begin{gather}
    m_{ij} \leq f(x, y) \leq M_{ij}, \ \forall (x, y) \in X_i \times Y_j
    \label{th561:uneq1}
  \end{gather}
  Фиксируем $\forall x \in X_i: x = \xi_i$. \\
  Учитывая \eqref{th561:uneq1} получим:
  \begin{gather*}
    m_{ij} |Y_j| \leq \int\limits_{Y_j} f(\xi_i, y) dy \leq M_{ij} |Y_j|
  \end{gather*}
  Просуммировав все по $j$ получим:
  \begin{gather*}
    \sum\limits_{j} m_{ij} |Y_j| \leq \mathcal{J}(\xi_i) = \int\limits_Y
    f(\xi_i, y) dy \leq \sum\limits_j M_{ij} |Y_j|
  \end{gather*}
  Отсюда имеем:
  \begin{gather}
    \sum\limits_{i} |X_i| \sum\limits_j m_{ij} |Y_j| \leq \sum\limits_{i}
    \mathcal{J}(\xi_i) |X_i| \leq \sum\limits_{i} |X_i| \sum\limits_{j} M_{ij}
    |Y_j|
    \label{th561:uneq2}
  \end{gather}
  По середине мы получили интегральную сумму для функции $\mathcal{J}(x)$.
  Крайние челны --- это суммы Дарбу $s(f, T), S(f, T)$ для кратного интеграла
  \eqref{th561:ex1}. \\
  Например $\sum\limits_{i, j} m_{ij} |X_i \times Y_j|$. Таким образом
  неравенство \eqref{th561:uneq2} перепишется в виде:
  \begin{gather}
    s(f, T) \leq \sum\limits_{i} \mathcal{J}(\xi_i) |X_i| \leq S(f, T)
    \label{th561:uneq3}
  \end{gather}
  Так как кратный интеграл \eqref{th561:ex1} существует по условию, то при
  $\lambda(T) \to 0$ обе суммы Дарбу неравенства \eqref{th561:uneq3} стремятся
  к этому интегралу, откуда
  \begin{gather*}
    \left|\lim\limits_{\lambda(T) \to 0} \sum\limits_{i} \mathcal{J}(\xi_i)
    |X_i| \right| = \int\limits_{X \times Y} f(x, y) dx dy
  \end{gather*}
  Левая часть этого равенства есть повторный интеграл:
  \begin{gather*}
    \int\limits_X \mathcal{J}(x) dx = \int\limits_X dx \int\limits_Y f(x, y) dy
  \end{gather*}
\end{proof}

\begin{remark}
  Применяя эту теорему несколько раз, можно свести вычисление по $k$-мерному
  промежутку к вычислению $k$ одномерных интегралов.
\end{remark}

\begin{consequence}
  Пусть $\mathbb{D} \subset Oxy$ --- область, ограниченная двумя кривыми
  $\mathcal{J} = \varphi(x), \mathcal{J} = \psi(x), (\varphi(x) \leq \psi(x))$
  и двумя прямыми $x = a$ и $x = b$. \\
  Тогда, если для $f(x, y)$ существует $\iint\limits_\mathbb{D} f(x, y)
  dx dy, \ x \in [a,b]$. \\
  $\mathcal{J}(x) = \int\limits_{\varphi(x)}^{\psi(x)} f(x, y) dy$, то
  существует так же повторный интеграл: $\int\limits_a^b dx
  \int\limits_{\varphi(x)}^{\psi(x)} f(x, y) dy$ и выполняется равенство
  \begin{gather*}
    \iint\limits_\mathbb{D} f(x, y) dx dy = \int\limits_a^b dx
    \int\limits_{\varphi(x)}^{\psi(x)} f(x, y) dy.
  \end{gather*}
\end{consequence}

\section{Замена переменных в кратных интегралах}
Напомним, что обтображение:
\begin{gather*}
  x = \varphi(t) =
  \begin{cases}
    x_1 = \varphi_1(t_1, \dots, t_n) \\
    \dots \\
    x_n = \varphi_n(t_1, \dots, t_n)
  \end{cases}
\end{gather*}
называется регулярным в области $\mathbb{D} \subset \mathbb{R}^n$, если:
\begin{enumerate}
  \item $\varphi_1, \dots, \varphi_n$ имеют в $\mathbb{D}$ непрерывные частные
    производные по всем аргументам.
  \item Матрица Якоби:

    \begin{gather}
      \varphi'(t) :=
      \begin{pmatrix}
        \frac{\partial \varphi_1}{\partial t_1} & \dots & \frac{\partial
        \varphi_1}{\partial t_n} \\
        \vdots & \ddots & \vdots \\
        \frac{\partial \varphi_n}{\partial t_1} & \dots & \frac{\partial
        \varphi_n}{\partial t_n}
      \end{pmatrix}
      \label{ch57:mat1}
    \end{gather}
    Определитель(якобиан):
    \begin{gather*}
      \det \varphi'(t) =
      \left|
      \begin{matrix}
        \frac{\partial \varphi_1}{\partial t_1} & \dots & \frac{\partial
        \varphi_1}{\partial t_n} \\
        \vdots & \ddots & \vdots \\
        \frac{\partial \varphi_n}{\partial t_1} & \dots & \frac{\partial
        \varphi_n}{\partial t_n}
      \end{matrix}
      \right|
      =: \frac{\mathbb{D}(\varphi_1, \dots, \varphi_n)}{\mathbb{D}(t_1, \dots,
      t_n)}
    \end{gather*}
    Отличен от нуля в $\mathbb{D}$.
    Матрица Якоби \eqref{ch57:mat1} называется производной отображения
    $\varphi$, а линейный оператор в $\mathbb{R}^n$ называется диффиренциалом
    $\varphi$ в точке $t$, в области $\mathbb{D}, \ (\mathbb{D}_\varphi(t))$.
    Регулярное отображение является локально обратимым.
\end{enumerate}
\begin{lemma}
  Пусть $\varphi$ --- регулярное отображение области $\mathbb{D} \subset
  \mathbb{R}_t^n$, $I$ --- замкнутый промежуток, лежащий в $\mathbb{D}, \ (I
  \subset \mathbb{D})$. $\varphi(I)$ --- образ промежутка $I$, $(\varphi(I)
  \subset R_x^n)$. Тогда, существует такая точка $\tau \in I$, что:
  \begin{gather*}
    \mu(\varphi(I)) = |\det \varphi'(\tau)| \cdot |I|
  \end{gather*}
  \begin{comment}
    $|I| = \mu(I)$
  \end{comment}
\end{lemma}

\begin{clarification}
  Лемма при $n = 1$ следует из формулы Лагранжа:
  $\varphi(b) - \varphi(a) = \varphi(\tau)(b-a)$, \\ где $\tau \in [a,b] = I$,
  поскольку при $\varphi'(\tau) \not = 0$ функция $\varphi$ монотонна и
  \begin{gather*}
    |\varphi(b) - \varphi(a)| = \mu(\varphi(I))
  \end{gather*}
  Если $n \geq 2$ и отображение $\varphi$ является линейным преобразованием
  \begin{gather*}
    x = \varphi(t) =
    \begin{cases}
      x_1 = a_{11}t_1 + \dots + a_{1n} t_n \\
      \dots \\
      x_n = a_{n1}t_1 + \dots + a_{nn} t_n \\
    \end{cases}
  \end{gather*}
  с матрицей $A = (a_{ij}) = \varphi'(t), \ x = At$, то образ промежутка
  $\varphi(I)$ является параллелепипедом $\varphi(I) \subset \mathbb{R}_x^n$,
  объем которого равен:
  \begin{gather*}
    |\det A| \cdot |I| = |\det \varphi'| \cdot |I|
  \end{gather*}
  При $n = 3, I$ --- прямоугольный параллелепипед, построенный на векторах:
  $h^1 = (h_1, 0, 0), h^2 = (0, h_2, 0), h^3 = (0, 0, h_3)$, тогда
  $\varphi(h^i) = (a_{1i} h_i, a_{2i} h_i, a_{3i} h_i), \ i = 1, 2, 3$. Отсюда
  объем параллелепипеда, построенного на векторах $\varphi(h^1), \varphi(h^2),
  \varphi(h^3)$ как на ребрах, равен
  \begin{gather*}
    |<\varphi(h^1), \varphi(h^2), \varphi(h^3)>| = |\det A| \cdot |h_1 h_2 h_3|
  \end{gather*}
  Рассматривая уже общий, нелинейный случай, следует принять во внимание, что в
  малой окрестности точки $t \in \mathbb{D}$ является почти линейным
  отображением:
  \begin{gather*}
    \varphi(t + h) = \varphi(t) + \mathbb{D}_\varphi(t) +
    \bar{\bar{o}}\left(h\right), \ \text{при} \ h \to 0
  \end{gather*}
  Поэтому, если размеры промежутка $I$ малы, то с малой относительной
  погрешностью можно сказать:
  \begin{gather*}
    \mu(\varphi(I)) \approx |\det \varphi'(t)| \cdot |I|, \ t \in I
  \end{gather*}
  В такой ситуации и используется лемма ($I$ --- малый промежуток).
\end{clarification}

\begin{theorem}
  Пусть $\varphi: \mathbb{E}_t \to \mathbb{E}_x$ --- отображение измеримого (по
  Жордану) множества $\mathbb{E}_t \subset \mathbb{R}_t^n, \ \mathbb{E}_x
  \subset \mathbb{R}_x^n$, при чем отображение $\varphi$ регулярно в некоторой
  области $\mathbb{D}$, содержащей замыкание $\overline{\mathbb{E}}_t,
  \mathbb{E}_t, (\overline{\mathbb{E}}_t \subset \mathbb{D})$. Тогда, если
  $f(x) \in \mathcal{R}(\mathbb{E}_x)$, то $f(\varphi(t)) |\det \varphi'(t)|
  \in \mathcal{R}(\mathbb{E}_t)$ и имеет место равенство:
  \begin{gather}
    \int\limits_{\mathbb{E}_x} f(x) dx = \int\limits_{\mathbb{E}_t}
    f(\varphi(t)) |\det \varphi'(t)| dt
    \label{th571:eq1}
  \end{gather}
\end{theorem}

\begin{proof}
  Случай, когда $\mathbb{E}_t$ --- промежуток, а $f(x)$ ограничена и непрерывна
  на $\mathbb{E}_x$. В этом случае функция $g(t) = f(\varphi(t))|\det
  \varphi'(t)|$ так же ограничена и непрерывна на $\mathbb{E}_t$, а
  следовательно и интегрируема на $\mathbb{E}_t$. Любому разбиению $T$
  промежутка $\mathbb{E}_t$ на промежутке $I_1, \dots, I_k$ соответствует
  разложение множества $\varphi(\mathbb{E}_t) = \mathbb{E}_x$ на множество
  $\varphi(I_j), \ j = 1, \dots, k$. \\
  Все эти множества измеримы, связаны и пересекаются попарно лишь по множествам
  меры нуль. Поэтому, в силу аддитивности интеграла:
  \begin{gather}
    \int\limits_{\mathbb{E}_x} f(x) dx = \sum\limits_{j = 1}^{k}
    \int\limits_{\varphi(I)} f(x) dx
    \label{th571:eq2}
  \end{gather}
  По теореме о среднем:
  \begin{gather}
    \int\limits_{\varphi(I_j)} f(x) dx = f(\xi^j) \mu(\varphi(I_j)), \ \xi^j
    \in \varphi(I_j)
    \label{th571:eq3}
  \end{gather}
  Пусть $\eta^j = \varphi^{-1}(\xi^j) \in I_j$, так что $\xi^j =
  \varphi(\eta^j)$. \\
  Поскольку по лемме $\mu(\varphi(I_j)) = |\det \varphi'(\tau^j)|\cdot |I_j|$,
  где $\tau^j \in I_j$, то из \eqref{th571:eq2} и \eqref{th571:eq3} получаем:
  \begin{gather}
    \int\limits_{\mathbb{E}_x} f(x) dx = \sum\limits_{j = 1}^{k}
    f(\varphi(\eta^j))\left|\det \varphi'(\tau^j)\right|\cdot |I_j| =: \sigma_1
    \label{th571:eq4}
  \end{gather}
  Составим сумму: $\sigma_2 := \sum\limits_{j = 1}^{k} f(\varphi(\eta^j))
  \left|\det \varphi'(\eta^j)\right| \cdot |I_j|$, которая является
  интегрируемой суммой для функции $g(t), \ t \in \mathbb{E}_t$. Поскольку
  $g(t) \in \mathcal{R}$, то:
  \begin{gather}
    \lim\limits_{\lambda(T) \to 0} \sigma_2 = \int\limits_{\mathbb{E}_t} g(t)
    dt
    \label{th571:eq5}
  \end{gather}
  Положим $\psi(t) = |\det \varphi'(t)|$ и оценим разность $\sigma_1 -
  \sigma_2$:
  \begin{gather*}
    \sigma_1 - \sigma_2 = \sum\limits_{j = 1}^{k} f(\xi^j) (\psi(\tau^j) -
    \psi(\eta^j)) |I_j|
  \end{gather*}
  Функция $\psi(t)$ непрерывна на замкнутом множестве $\mathbb{E}_t$ и по
  теореме Кантора равномерно непрерывна на $\mathbb{E}_t$, так что $\forall
  \varepsilon > 0 \ \exists \delta > 0, \ \lambda(T) < \delta$:
  \begin{gather*}
    |\psi(\tau^j) - \psi(\eta^j)| < \varepsilon, \ \forall j \\
    \rho(\tau^j, \eta^j) < \delta
  \end{gather*}
  Отсюда, при $\lambda(T) < \delta$:
  \begin{gather*}
    |\sigma_1 - \sigma_2| \leq \max\limits_{x \in \mathbb{E}_x} |f(x)| \cdot
    \varepsilon \sum\limits_{j = 1}^{k} |I_j| = C \varepsilon
  \end{gather*}
  где $C$ не зависит от $\varepsilon$, $C = const$.
  \begin{gather}
    \lim\limits_{\lambda(T) \to 0} (\sigma_1 - \sigma_2) = 0
    \label{th571:eq6}
  \end{gather}
  Равенство \eqref{th571:eq1} теперь следует из \eqref{th571:eq4},
  \eqref{th571:eq5} и \eqref{th571:eq6}. \\
  Доказательство теоремы в общем случае можно провести, придерживаясь
  рассмотренной схемы.
\end{proof}

\begin{consequence}
  Величина интеграла от $f$ по множеству $\mathbb{E} \subset \mathbb{R}^n$ и
  зависит от выбора декартовых координат в $\mathbb{R}^n$.
\end{consequence}

\begin{proof}
  Пусть $\mathbb{E}_x, \mathbb{E}_t$ --- запись множества $\mathbb{E}$.
  $p$ --- точка множества $\mathbb{E}, \ x = (x_1, \dots, x_n)$ --- ее
  координаты в первой системе, $t = (t_1, \dots, t_n)$ --- во второй системе.
  Тогда $f(p) = f_x(x_1, \dots, x_n) = f_t(t_1, \dots, t_n)$, где $f_t = f_x
  \circ \varphi$ (суперпозиция). Поскольку переход от одной системы декартовых
  координат к другой имеет якобиан по модулю равный единице, то есть:
  \begin{gather*}
    \int\limits_{\mathbb{E}_x} f_x(x) dx = \int\limits_{\mathbb{E}_t}
    f_x(\varphi(t))\left|\det \varphi'(t)\right| dt = \int\limits_{\mathbb{E}_t} f_t(t) dt
  \end{gather*}
\end{proof}

