\section{Криволинейный интеграл первого рода}
Пусть $\Gamma = \{M(S) : 0 \leq S \leq S_0\}$ определенная кривая в
$\mathbb{R}^3$, в частности $\mathbb{R}^2. \\ M(S) = (x(S), y(S), z(S))$ --- ее
параметрическое представление, где в качестве переменной взята длина дуги $S$.
\\
Кривая, у которой переменная длина дуги отсчитывается от точки $A$
обозначим: $\Gamma = \wideparen{AB}$. Противоположно ориентированную прямую, у
которой переменная длина дуги отсчитывается от точки $B$ обозначим $\Gamma =
\wideparen{BA}$.

\begin{definition}
  Пусть на токах $M(S)$, кривой $\Gamma = \wideparen{AB}$ задана некотороая
  функция $F$. Выражение $\int\limits_\Gamma F(x, y, z) dS$ опеределенное по
  формуле:
  \begin{gather}
    \int\limits_\Gamma F(x, y, z) dS = \int\limits_0^{S_0} F(x(S), y(S), z(S))
    dS
    \label{def611:eq1}
  \end{gather}
  называется криволинейным интегралом первого рода (КИПР) от функции $F$ по
  кривой $\Gamma$. Этот интеграл обозначается символами:
  \begin{gather*}
    \int\limits_\Gamma F(M(S)) dS, \ \int\limits_\Gamma F dS, \
    \int\limits_{\wideparen{AB}} FdS
  \end{gather*}
\end{definition}

Отметим свойства интеграла \eqref{def611:eq1}:
\begin{enumerate}
  \item $\int\limits_\Gamma dS = S_0$ --- длина кривой $\Gamma$.
  \item Если $F$ непрерывна на $\Gamma$ (то есть $F(x(S), y(S), z(S))$
    непрерывная на отрезке $S \in [0, S_0]$), то $\int\limits_\Gamma FdS$
    существует.
  \item КИПР не зависит от ориентации кривой:
    \begin{gather*}
      \int\limits_{\wideparen{BA}} FdS = \int\limits_{\wideparen{AB}} FdS
    \end{gather*}
\end{enumerate}

\begin{definition}
  Кривую $\Gamma$ называют гладкой, если в ее параметрическом представлении:
  \begin{gather}
    x = \varphi(t), \ y = \psi(t), \ z = \chi(t), \ a \leq t \leq b
    \label{def612:eq2}
  \end{gather}
  все функции непрерывно диффиренцируемы на отрезке $[a, b]$ и $\varphi'(t)^2 +
  \psi'(t)^2 + \chi'(t)^2~>~0, \ \forall t \in~[a, b]$.
\end{definition}

\begin{theorem}
  Пусть $\Gamma$ --- гладкая кривая с параметрическим представлением
  \eqref{def612:eq2},\\ $F$ --- непрерывная на $\Gamma$ функция. Тогда:
  \begin{gather}
    \int\limits_\Gamma F(x, y, z) dS = \int\limits_a^b F(\varphi(t), \psi(t),
    \chi(t)) \sqrt{\varphi'(t)^2 + \psi'(t)^2 + \chi'(t)^2} dt
    \label{th611:eq1}
  \end{gather}
\end{theorem}

\begin{proof}
  Кривая $\Gamma$ спрямляема. Переменную дуги $S = S(t)$ можно принять в
  качестве параметра, и тогда $\int\limits_\Gamma FdS$ существует. Учитывая,
  что $S'(t) = \sqrt{\varphi'(t)^2 + \psi'(t)^2 + \chi'(t)^2}$ получим
  \eqref{th611:eq1}.
\end{proof}

\begin{remark}
  Если плоская кривая $\Gamma$ является графиком функци $y = f(x), a \leq x
  \leq b$, то ее представлением является функция $x = x, y = f(x)$. И тогда
  формула \eqref{th611:eq1} примет вид:
  \begin{gather*}
    \int\limits_\Gamma F(x, y) dS = \int\limits_a^b F(x, f(x)) \sqrt{1 +
    f'(x)^2} dx
  \end{gather*}
\end{remark}

\section{Криволинейный интеграл второго рода}

Пусть $\Gamma$ --- гладкая ориентированная кривая,\\ $\vec{r} = \vec{r}(S) =
\{x(S), y(S), z(S)\}, \ 0 \leq S \leq S_0$ --- ее векторное представление, в
котором за параметр $S$ взята переменная длина ее дуги. $\vec{\tau} =
\frac{d \vec{r}}{dS} = \{\frac{dx}{dS}, \frac{dy}{dS}, \frac{dz}{dS}\}$ ---
единичный касательный вектор к кривой $\Gamma$, его направление соответствует
выбранному направлению отсчета длины дуг. Если $\alpha, \beta, \gamma$ ---
углы, которые $\vec\tau$ образует с координатными осями, то $\vec\tau = \{\cos
\alpha, \cos \beta, \cos \gamma\}$. Получаем $\cos \alpha~=~\frac{dx}{dS}, \cos
\beta~=~\frac{dy}{dS}, \cos \gamma~=~\frac{dz}{dS}$. \\

Пусть на $\Gamma$ задана вектор-функция $\vec a = \vec a(x, y, z) = \vec
a(x(S), y(S), z(S)), \ 0 \leq S \leq S_0$. \\ Пусть $P, Q, R$ --- координаты
вектора $\vec a : \vec a = \{P, Q, R\}$. Функции $P, Q, R$ являются функциями
точки кривой $\Gamma$.

\begin{definition}
  \label{def621}
  $\int\limits_\Gamma \vec a d \vec r$ определенное по формуле:
  \begin{gather}
    \int\limits_\Gamma (\vec a, \vec \tau) dS
    \label{def621:int1}
  \end{gather}
  называется криволинейным интегралом второго рода (КИВР) по кривой $\Gamma =
  \wideparen{AB}$.
\end{definition}

Для интеграла $\int\limits_\Gamma \vec a d \vec r$ используется так же
обозначение:
\begin{gather*}
  \int\limits_\Gamma Pdx + Qdy + Rdz
\end{gather*}
где $Pdx + Qdy + Rdz$ --- записанное в координатной форме скалярное
произведение векторов $\vec a = \{P, Q, R\}$ и $d \vec r = \{dx, dy, dz\}$.
Учитывая, что $(\vec a, \vec \tau) = P \cos \alpha + Q \cos \beta + R \cos
\gamma$, то определение \eqref{def621} можно записать в виде:
\begin{gather}
  \int\limits_\Gamma \vec a d \vec r \equiv \int\limits_\Gamma Pdx + Qdy + Rdz
  := \int\limits_\Gamma (P\cos \alpha + Q \cos \beta + R \cos \gamma) dS
  \label{def621:int2}
\end{gather}

Следующие свойства КИВР под силу доказать каждому:
\begin{enumerate}
  \item $\vec a = \vec a(x, y, z), \ \Gamma = \wideparen{AB}$, то
    $\int\limits_\Gamma \vec a d \vec r$ существует.
  \item КИВР меняет знак при изменении ориентации кривой $\Gamma$:
    \begin{gather*}
      \int\limits_{\wideparen{BA}} \vec a d \vec r =
      -\int\limits_{\wideparen{AB}} \vec a d \vec r
    \end{gather*}
\end{enumerate}

\begin{theorem}
  Если $\Gamma = \wideparen{AB}$ --- гладкая ориентированная кривая. \\ $\vec r =
  \vec r(t) = \{\varphi(t), \psi(t), \chi(t)\}, \ a \leq t \leq b$ --- ее
  векторное представление ($\vec r(a) = A, \ \vec r(b) = B$), то получим:
  \begin{gather}
    \int\limits_\Gamma \vec a d \vec r = \int\limits_a^b (\vec a, \vec r') dt
    \label{th621:eq1}
  \end{gather}
\end{theorem}

\begin{proof}
  Поскольку
  \begin{gather}
    \vec \tau = \frac{d \vec r}{dS} = \frac{d \vec r}{dt} \cdot
    \frac{dt}{dS} = \frac{\vec r'}{S'}
    \label{th621:eq2}
  \end{gather}
  (здесь штрихом обозначены производные по $t$), то
  \begin{gather*}
    \int\limits_\Gamma \vec a d \vec r \overset{\eqref{def621:int1}} =
    \int\limits_\Gamma (\vec a, \vec \tau) dS \overset{\eqref{th611:eq1}} =
    \int\limits_a^b (\vec a, \vec \tau) S' dt \overset{\eqref{th621:eq2}} =
    \int\limits_a^b (\vec a, \vec r') dt
  \end{gather*}
\end{proof}

\begin{remark}
  В координатной форме формула \eqref{th621:eq1} примет вид:
  \begin{gather}
    \int\limits_\Gamma Pdx + Qdy + Rdz = \int\limits_a^b (P x' + Q y' + R z')dt
    \label{rem621:eq1}
  \end{gather}
\end{remark}

\begin{consequence}
  Если плоская кривая $\Gamma = \wideparen{AB}$ является графиком функции $y =
  f(x), \ a~\leq~x~\leq~b, \\ A = (a, f(a)), \ B = (b, f(b))$, то формула
  \eqref{th621:eq1} (при $x = t, \ R \equiv 0$) принимает вид:
  \begin{gather}
  \int\limits_\Gamma Pdx + Qdy + Rdz = \int\limits_a^b (P(x, f(x)) + Q(x,
  f(x))) f'(x) dx
  \label{cons621:eq1}
  \end{gather}
\end{consequence}

\begin{remark}
  Если $\vec a = \{P, 0, 0\}, \ Q \equiv R \equiv 0$, то $\int\limits_\Gamma
  \vec a d \vec r$ обозначается $\int\limits_\Gamma dx$, таким образом
  интеграл:
  \begin{gather}
    \int\limits_\Gamma P(x, y, z) dx = \int\limits_\Gamma P \cos \alpha \ dS
    \label{rem622:eq1}
  \end{gather}
  аналогично
  \begin{gather}
    \int\limits_\Gamma Q dy := \int\limits_\Gamma Q \cos \beta \ dS,
    \ \int\limits_\Gamma R dz := \int\limits_\Gamma R \cos \gamma \ dS
    \label{rem622:eq2}
  \end{gather}
  Отсюда, воспользовавшись адитивностью обычного интеграла, получим:
  \begin{gather*}
    \int\limits_\Gamma Pdx + Qdy + Rdz = \int\limits_\Gamma Pdx +
    \int\limits_\Gamma Qdy + \int\limits_\Gamma Rdz
    \label{rem622:eq3}
  \end{gather*}
\end{remark}

\begin{consequence}
  В условиях предыдущего следствия:
  \begin{gather}
    \int\limits_\Gamma Pdx = \int\limits_a^b P(x, f(x)) dx
    \label{cons621:eq2}
  \end{gather}
\end{consequence}

\begin{definition}
  Если кривая $\Gamma$ --- кусочно-гладкая, то есть представима в виде
  объединения конечного числа гладких ориентированных кривых $\Gamma_1, \dots,
  \Gamma_k, \ F, \ \vec a = \{P, Q, R\}$, определенный на точках кривой
  $\Gamma$, то пологают:
  \begin{gather*}
    \int\limits_\Gamma F dS := \sum\limits_{i = 1}^{k} \int\limits_{\Gamma_i} F
    dS, \int\limits_\Gamma \vec a d \vec r := \sum\limits_{i = 1}^{k}
    \int\limits_{\Gamma_i} \vec a d \vec r
  \end{gather*}
\end{definition}

\section{Формула Грина}
Пусть на плоскости $\mathbb{R}^2$ задана система координат по $Oxy$.

\begin{definition}
  Ориентация простого замкнутого контура $\Gamma$, лежащего на этой плоскости
  называется положительной, если она соответствует движению {\bfseries против}
  часовой стрелки. Противоположная ориентация называется отрицательной.
\end{definition}

Напомним, что простым замкнутым контуром в $\mathbb{R}^n$ называют $x =
\varphi(t) = (\varphi_1(t), \dots, \varphi_n(t)), \\ a \leq t \leq b$, у
которой нет других кратных точек.

\begin{definition}
  Пусть граница $\partial \mathbb{G}$, ограниченной плоской области
  $\mathbb{G}$,
  состоит из конечного числа простых замкнутых контуров. Совокупность этих
  контуров, ориентированных так, что при обходе каждого из них, область
  $\mathbb{G}$ остается слева(справа), называется положительной(отрицательной)
  ориентацией: $\partial \mathbb{G}^+ (\partial \mathbb{G}^-)$.
\end{definition}

\begin{definition}
  Грацница $\partial \mathbb{G}$ области $\mathbb{G}$ называется
  кусочно-гладкой, если она состоит из конечного числа простых кусочно-гладких
  контуров.
\end{definition}

\begin{remark}
  Если граница $\partial \mathbb{G}$ области $\mathbb{G}$ является кусочно
  гладкой, то ее площадь равна нулю, а само множество $\mathbb{G}$ квадрируемо.
\end{remark}

\begin{theorem}
  \label{th631}
  Если граница плоской, ограниченной области $\mathbb{G}$ является
  кусочно-гладкой, а функции $P, Q, \frac{\partial P}{\partial y},
  \frac{\partial Q}{\partial x}$ --- непрерывны на замыкании
  $\overline{\mathbb{G}}$ области $\mathbb{G}$, то:
  \begin{gather}
    \iint\limits_{\mathbb{G}} \left(\frac{\partial Q}{\partial x} -
    \frac{\partial P}{\partial y} \right) dx dy = \int\limits_{\partial
      \mathbb{G}^+} Pdx + Qdy
    \label{th631:eq1}
  \end{gather}
  эта формула называется {\bfseries формулой Грина}.
\end{theorem}

\begin{proof}
  Проведем его для произвольных областей.
\end{proof}

\begin{definition}
  Назовем область $\mathbb{G}$ областью $\mathbb{G}_y$, если $\mathbb{G}$ имеет
  вид:
  \begin{gather}
    \mathbb{G} = \{(x, y) : a < x < b, \ \varphi(x) < y < \psi(x)\}
    \label{def634:eq1}
  \end{gather}
  где $\varphi(x), \psi(x)$ --- кусочно-гладкие функции на $[a,b]$. Поменяв
  здесь $x$ и $y$ ролями получим определение области $\mathbb{G}_x$. Области,
  которые можно разрезать на куски вида $\mathbb{G}_x$ либо $\mathbb{G}_y$
  назовем простыми областями.
\end{definition}

\begin{lemma}
  Пусть в теореме \eqref{th631} $\mathbb{G}$ --- область по $\mathbb{G}_y$,
  тогда:
  \begin{gather}
    \iint\limits_{\mathbb{G}} \frac{\partial P}{\partial y} dx dy = -
    \int\limits_{\partial \mathbb{G}^+} Pdx
    \label{lem631:eq1}
  \end{gather}
\end{lemma}

\begin{proof}
  Сводя двойной интеграл к повторному, применяя формулу Ньютона-Лейбница и
  формулу \eqref{cons621:eq2} получим:
  \begin{gather}
    \iint\limits_{\mathbb{G}} \frac{\partial P}{\partial y} dx dy =
    \int\limits_a^b dx \int\limits_{\varphi(x)}^{\psi(x)} \frac{\partial
    P}{\partial y} dy = \int\limits_a^b \left[P(x, \psi(x)) - P(x,
    \varphi(x))\right] dx = \\
    -\int\limits_a^b P(x, \varphi(x)) dx - \int\limits_b^a P(x, \psi(x)) dx =
    -\int\limits_{\wideparen{AB}} P(x, y) dx - \int\limits_{\wideparen{CD}}
    P(x, y) dx
    \label{lem631:eq2}
  \end{gather}
  заметим, что $\int\limits_{\wideparen{BC}} Pdx = \int\limits_{\wideparen{DA}}
  Pdx = 0$, и что сумма: $\int\limits_{\wideparen{AB}} Pdx +
  \int\limits_{\wideparen{BC}} Pdx + \int\limits_{\wideparen{CD}} Pdx +
  \int\limits_{\wideparen{DA}} Pdx = \int\limits_{\partial \mathbb{G}^+} Pdx$.
  Поэтому из \eqref{lem631:eq2} получаем \eqref{lem631:eq1}. \\
  Аналогично, если $\mathbb{G}$ --- область типа $\mathbb{G}_x$, то:
  \begin{gather}
    \iint\limits_{\mathbb{G}} \frac{\partial Q}{\partial x} dx dy =
    \int\limits_{\partial \mathbb{G}^+} Qdy
    \label{lem631:eq3}
  \end{gather}
\end{proof}

\begin{lemma}
  Если в теореме \eqref{th631} область $\mathbb{G}$ допускает разбиение на
  области $\mathbb{G}_x, \mathbb{G}_y$, то выполняется \eqref{lem631:eq1}.
\end{lemma}

\begin{proof}
  Двойной интеграл в области $\mathbb{G}$ в силу адитивности есть сумма по
  кускам $\mathbb{G}_x, \mathbb{G}_y$, на которые разрезана $\mathbb{G}$. Для
  каждого справедливо \eqref{lem631:eq1}. Но соседние куски на общей части
  границ индуцируют противоположные ориентации. Поэтому, при сложении
  интегралов по границам всех кусков в результате взаимных уничтожений,
  останется $\int\limits_{\mathbb{G}} \partial \mathbb{G}$. \\
  Аналогично, если разрезается на $\mathbb{G}_x$, то справедливо
  \eqref{lem631:eq3}. Запищем для произвольной области выражения
  \eqref{lem631:eq1} и \eqref{lem631:eq3}, умножив \eqref{lem631:eq1} на $-1$.
  После сложения полученных соотношений, пользуясь свойством адитивности
  интеграла, получим формулу Грина.
\end{proof}

\begin{consequence}
  \begin{gather*}
    \mu(\mathbb{G}) = \iint\limits_{\mathbb{G}} dx dy \overset{(\rom{1})} =
    \int\limits_{\partial \mathbb{G}^+} x dy \overset{(\rom{2})} =
    -\int\limits_{\partial \mathbb{G}^+} y dx \overset{(\rom{3})} =
    \frac{1}{2} \int\limits_{\partial \mathbb{G}^+} xdy - ydx
  \end{gather*}
  \begin{align*}
    (\rom{1}) &\ \text{---} \ P = 0, Q = x \\
    (\rom{2}) &\ \text{---} \ P = -y, Q = 0 \\
    (\rom{3}) &\ \text{--- если возьмем полусумму}
  \end{align*}
\end{consequence}


