\section{Радиус сходимости и круг сходимости}
\begin{definition}
  Степенной ряд --- ряд вида
  \begin{gather}
    \sum\limits_{n = 0}^{\infty} a_n(z - z_0)^n, \ z, z_0 \in \mathbb{C}, n = 0, 1, \dots
    \label{def311:series1}
  \end{gather}
  $a_n$ --- коэффициенты ряда. \\
  $\xi = z - z_0, $ тогда
  $\sum\limits_{n = 0}^{\infty} a_n \xi$,
  \begin{gather}
    \sum\limits_{n = 0}^{\infty} a_n z^n \label{def311:series2}
  \end{gather}
\end{definition}

\begin{theorem}
  \label{th311}
  Степенной ряд \ref{def311:series2}, $\alpha = \overline{\lim} \sqrt[n]{|a_n|}$,
  \begin{gather}
    R = \frac{1}{\alpha} \label{def311:eq1}
  \end{gather}
  ($\alpha = 0 \Longleftrightarrow R = \infty, \
  \alpha = +\infty \Longleftrightarrow, R = 0$), тогда ряд \ref{def311:series2}
  абсолютно сходится, если $|z| < R,$ и рассходится, если $|z| > R$.
\end{theorem}

\begin{proof}
  Положим $C_n = a_n z^n$. По критерию Коши заключаем, что сумма
  $\sum C_n$ сходится при $\overline{\lim\limits_{n \to \infty}} \sqrt[n]{|a_n|}
  = |z|\cdot \overline{\lim\limits_{n \to \infty}} \sqrt[n]{|a_n|} =
  \frac{|z|}{R} < 1,$ то есть $|z| < R;$ и рассходится, если $|z| > R$.
\end{proof}

\begin{definition}
  Число R называется радиусом сходимости ряда \ref{def311:series2}. \\
  $|z| < R, z \in \mathbb{C}$ называется кругом сходимости ряда
  \ref{def311:series2}.
\end{definition}

\begin{remark}
  О сходимсоти на границе окружности $|z| = R$ ничего не говорится в теореме
  \ref{th311}, так как возможны все варианты.
\end{remark}

\begin{theorem}
  Если R --- радиус сходимости $(R > 0)$ ряда \ref{def311:series2}, то на любом
  круге $|z| < r, $ где $r$ --- фиксированно, и $r < R$. \\
  Таким образом этот ряд сходится абсолютно и равномерно.
\end{theorem}

\begin{proof}
  $z = r, \sum\limits_{n = 0}^{\infty} |a_n| r^n$ сходится, а так как для любой
  точки $z$ круга $|z| \leq r$ выполняется неравенство:
  \begin{gather*}
    |a_n z^n| \leq |a_n| r^n, \ \forall n
  \end{gather*}
  то по признаку Вейерштрассе на этом круге ряд \ref{def311:series2} сходится
  равномерно.
\end{proof}

\begin{consequence}
  Степеной ряд непрерывный в каждой точке своего круга $|z| < R$ сходится.
\end{consequence}

\begin{theorem}[2-ая т. Абеля]
  Если R --- радиус сходимости, $\sum\limits_{n = 0}^{\infty} a_n z^n$ и этот
  ряд сходится при $|z| = R,$ то он сходится на отрезке $[0, R]$ равномерно.
\end{theorem}

\begin{proof}
  Пусть $0 \leq x \leq R$, представим ряд $\sum\limits_{n = 0}^{\infty} a_n x^n
  = \sum\limits_{n = 0}^{\infty} a_n R^n\left(\frac{x}{R}\right)^n$. По скольку
  члены ряда $\sum a_n R^n$ не зависит от $x$, то его сходимость означает его
  равномерную сходимость. $\{(\frac{x}{R})^n\}$ ограничена на отрезке $[0, R]$
  и монотонна в каждой точке. \\
  Поэтому в силу признака Абеля равномерной сходимости рядов \ref{th232} ряд
  \ref{def311:series2} равномерно сходится на отрезке $[0, R]$.
\end{proof}

\begin{lemma}
  Радиусы сходимости $R, R_1, R_2$ соответственно рядов
  $\sum\limits_{n = 0}^{\infty} a_n z^n, \sum\limits_{n = 0}^{\infty}
  \frac{a_n}{n + 1} z^{n+1}, \sum\limits_{n = 0}^{\infty} n a_n z^{n - 1}$ равны:
  $R = R_1 = R_2$.
\end{lemma}

\begin{proof}
  Действительно, так как $\lim\limits_{n \to \infty} \sqrt[n]{\frac{1}{n + 1}}=
  \lim\limits_{n \to \infty} \sqrt[n]{n} = 1$, то \\
  $\overline{\lim\limits_{n \to \infty}} \sqrt[n]{|a_n|} = 
  \overline{\lim\limits_{n \to \infty}} \sqrt[n]{\frac{a_n}{n + 1}}=
  \overline{\lim\limits_{n \to \infty}} \sqrt[n]{|n a_n|}$
\end{proof}
