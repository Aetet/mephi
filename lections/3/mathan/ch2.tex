\section{Поточечная сходимость}
Пусть на некотором множестве $\mathbb{E}$ задана последовательность
комплексно значимых функций $f_n, n = 1, 2, \dots \ , \ (f_n \in \mathbb{C})$.
Элементы $x \in \mathbb{E}$ будем называть точками.
\begin{definition}
  $\{f_n\}$ называется ограниченной на $\mathbb{E}$, если
  $\exists M > 0: \forall n \in \mathbb{N}, \forall x \in \mathbb{E}$ выполняется
  $$|f_n(x)| \leq M$$
\end{definition}

\begin{definition}
  $\{f_n\}$ называется сходящейся поточечно на множестве $\mathbb{E}$, если
  при любом фиксированном $x \in \mathbb{E}$, числовая последовательность
  $\{f_n(x)\}$ сходится. Если последовательность сходится на $\mathbb{E}$,
  то $f(x) := \lim\limits_{n \to \infty} f_n(x), x \in \mathbb{E}$ называется
  пределом последовательности. Пусть
  $\{U_n(x)\}_{n = 1}^\infty, x \in \mathbb{E}, \ (U_n \in \mathbb{C})$
  --- последовательность числовых функций.
\end{definition}

\begin{definition}
  Множество числовых рядов
  \begin{gather}
    \sum\limits_{n = 1}^{\infty} U_n(x) \label{def213:series1}
  \end{gather}
  в каждой из которых точка $x$ фиксированная называется рядом на множестве
  $\mathbb{E}$, а функция $U_n(x)$ --- его член. \\
  $S_n(x) = \sum\limits_{k = 1}^{n} U_k(x), x \in \mathbb{E}$ называется
  $n$-ой частичной суммой ряда \eqref{def213:series1}. \\
  $\sum\limits_{k = n + 1}^{\infty} U_k(x)$ - его $n$-ым остатком.
\end{definition}

\begin{definition}
  Ряд \eqref{def213:series1} называется сходящимся поточечно на множестве
  $\mathbb{E}$, если последовательность $\{S_n(x)\}$ сходится поточечно на
  $\mathbb{E}$. При этом $\lim\limits_{n \to \infty} S_n(x) = S(x), x \in \mathbb{E}$
  называется суммой ряда \eqref{def213:series1}.
  $$S(x) = \sum\limits_{n = 1}^{\infty} U_n(x).$$
\end{definition}

\begin{definition}
  Если ряд \eqref{def213:series1} при любом $x \in \mathbb{E}$ сходится абсолютно,
  то он называется абсолютно сходящимся на множестве $\mathbb{E}$.
\end{definition}

\begin{remark}
  Беззаботная перестановка членов ряда может привести к ошибке.
\end{remark}

\section{Равномерная сходимость}
\begin{definition}
  \label{def221}
  Говорят, что функциональная последовательность $\{f_n\}_{n=1}^\infty$
  сходится равномерно на $\mathbb{E}$, если $\forall \varepsilon > 0 \
  \exists N \in \mathbb{N}: \forall n > N, \forall x \in \mathbb{E}$
  имеем
  $$|f_n(x) - f(x)| < \varepsilon$$
  Ясно, что каждая равномерно сходящаяся последовательность, сходится поточечно.
\end{definition}

\begin{comment}
  Обозначение равномерной сходимости:
  $f_n \stackrel{\mathrm{\mathbb{E}}}{\rightrightarrows} f$
\end{comment}

\begin{theorem}[Критерий Коши равномерной сходимости последовательностей]
  \label{th221}
  Для того, чтобы $\{f_n\}$ равномерно сходилась на
  $\mathbb{E} \Longleftrightarrow$
  $\forall \varepsilon > 0 \ \exists N: n, m > N, \forall x \in \mathbb{E}:$
  \begin{gather}
    |f_n(x) - f_m(x)| < \varepsilon \label{th221:uneq1}
  \end{gather}
\end{theorem}

\begin{proof}
  \hfill
  \begin{itemize}
    \item Необходимость: \\
      $f_n \stackrel{\mathrm{\mathbb{E}}}{\rightrightarrows} f$, тогда
      $\forall \varepsilon > 0, \ \exists N \in \mathbb{N}: \forall n > N,
      \forall x \in \mathbb{E} \ |f_n(x) - f(x)| < \frac{\varepsilon}{2}$. \\
      $|f_n(x) - f_m(x)| \leq |f_n(x) - f(x)| + |f(x) - f_m(x)| <
      \frac{\varepsilon}{2} + \frac{\varepsilon}{2} = \varepsilon$,
      $(\forall n, m > N, \forall x \in \mathbb{E}).$
    \item Достаточность: \\
      Пусть выполняется условие Коши, тогда $\{f_n(x)\}$, удовлетворяет
      критерию Коши сходимости числовых последовательностей и следовательно
      сходящегося числового предела, который обозначим $f(x)$. \\
      Тогда перейдя к пределу при $m \to \infty$ получим $\forall n > N,
      \forall x \in \mathbb{E}: |f_n(x) - f(x)| < \varepsilon$. \\
  \end{itemize}
\end{proof}
Иногда полезен критерий, следующий из определения \eqref{def221}

\begin{theorem}
  Пусть $\lim\limits_{n \to \infty} f_n(x) = f(x), \forall x \in \mathbb{E}$. \\
  Положим $r_n = \sup|f_n(x) - f(x)|, x \in \mathbb{E}$ --- равномерное уклонение. \\
  Тогда $f_n \stackrel{\mathrm{\mathbb{E}}}{\rightrightarrows} f
  \Longleftrightarrow r_n \to 0, \ n \to \infty$. (Переформулировка определения).
\end{theorem}

\begin{proof}
  Без доказательства. \\
\end{proof}

\begin{example}
  $f_n(x) = x^n, \mathbb{E} = [0, 1)$ \\
  $\lim\limits_{n \to \infty} f_n(x) = 0, \forall \in \mathbb{E},
  r_n = \sup\limits_{x \in [0, 1)} |x^n - 0| = 1 \not \to 0, n \to \infty.$ \\
  $\{x^n\}$ не является равномерно сходящейся на $\mathbb{E}$.
\end{example}

\begin{example}
  $f_n(x) = x^n - x^{n+1}, \mathbb{E} = [0, 1]$. \\
  $f_n(x) \to 0, \forall x \in \mathbb{E}, \  f_n'(x) = nx^{n-1} - (n + 1)x^n = 0$. \\
  $x_n = \frac{n}{n+1}, \ f_n(x_n) = x_n^n(1 - x_n) < \frac{1}{n + 1}$. \\
  $r_n < \frac{1}{n + 1}$. \\
\end{example}

\begin{definition}
  \label{def222}
  \begin{gather}
    \sum\limits_{n = 1}^{\infty} U_n(x), \ x \in \mathbb{E} \label{def222:series1}
  \end{gather}
  называется равномерно сходящейся, если на множестве $\mathbb{E}$ равномерно
  сходится последовательность частичных сумм. \\
\end{definition}

Пусть $S_k(x)$ --- частичные $k$-ые суммы ряда \eqref{def222:series1},
$$m \geq n: U_n(x) + \dots + U_m(x) = S_m(x) - S_n(x)$$
тогда из теоремы \eqref{th221} (критерий Коши равномерной сходимости последовательности)
$\Rightarrow$ Теорема \eqref{th223} (критерий Коши равномерной сходимости ряда).

\begin{theorem}[Критерий Коши равномерной сходимости ряда]
  \label{th223}
  Для того, чтобы ряд \eqref{def222:series1} равномерно сходился на множестве
  $\mathbb{E} \Longleftrightarrow \forall \varepsilon > 0 \ \exists N \in \mathbb{N},
  \forall n,m > N,
  \forall x \in \mathbb{E}: $
  \begin{gather}
    |U_n(x) + \dots + U_m(x)| < \varepsilon \label{th223:uneq1}
  \end{gather}
\end{theorem}

\begin{proof}
  Без доказательства.
\end{proof}

\begin{consequence}[Необходимый признак равномерной сходимости]
  У равномерно сходящегося ряда общий член равномерно стремится к нулю.
\end{consequence}

\begin{theorem}[Признак Вейерштрасса]
  Пусть $\{U_n\}$ --- последовательнсоть функций, определенных на $\mathbb{E}$
  и пусть $|U_n(x)| \leq a_n, \forall x \in \mathbb{E}, \forall n \in \mathbb{N}.$
  Тогда если $\sum a_n < \infty$ сходится, то следовательно $\sum U_n(x)$ сходится
  равномерно на $\mathbb{E}$.
\end{theorem}
\begin{proof}
  Если $\sum a_n$ сходится, то $\forall \varepsilon > 0 |
  \sum\limits_{k = n}^{m} U_k(x)| \leq
  \sum\limits_{k = n}^{m} a_k < \varepsilon$, при любом $x \in \mathbb{E}$,
  если только $m$ и $n$ достаточно велики, теорема \eqref{th111} (критерий Коши
  сходимости числового ряда). Равномерная сходимость нашего ряда вытекает из
  теоремы \eqref{th223}.
\end{proof}
\begin{remark}
  $\sum a_n$ называется мажорирующим рядом $\sum U_n(x)$.
\end{remark}

\begin{remark}
  ПРОВЕРИТЬ!!! \\
  Условие признака Вейерштрасса не являются необходимыми для равномерной
  сходимости ряда.
\end{remark}

\section{Признаки равномерной сходимости рядов Дирихле и Абеля}
\begin{theorem}
  Пусть дан ряд
  \begin{gather}
    \sum\limits_{n = 1}^{\infty} a_n(x) b_n(x), \ x \in \mathbb{E} \label{th231:series1}
  \end{gather}
  такой что:
  \begin{enumerate}
    \item $a_n(x) \in \mathbb{R}, \ b_n(x) \in \mathbb{C}, \ n = 1, 2, \dots$
    \item $a_n(x) \stackrel{\mathrm{\mathbb{E}}}{\rightrightarrows} 0$
      (Равномерная сходимость к нулю), $\{a_n(x)\}$ - монотонна.
    \item $\{b_n(x)\}, \ \sum b_n(x)$ ограничена на множестве $\mathbb{E}$.
  \end{enumerate}
  Тогда ряд \eqref{th231:series1} равномерно сходится на множестве $\mathbb{E}$.
\end{theorem}

\begin{proof}
  В силу условия 3, $\exists B > 0: |B_n(x)| \leq B, \ \forall x \in \mathbb{E}, \
  \forall n \in \mathbb{N}$. \\
  $\forall x \in \mathbb{E}, m \geq n \geq 2: |b_n(x) + \dots + b_m(x)| =
  |B_m(x) - B_{n-1}(x)| \leq 2B$. \\
  $\forall \varepsilon > 0$ из условия 2 $\Rightarrow \exists N = N(\varepsilon):
  n > N(\varepsilon), \forall \in \mathbb{E}$ выполняется неравенство:
  $$0 \leq |a_n(x)| < \frac{\varepsilon}{6B}.$$
  Примениев лемму Абеля \eqref{th151:cons}, получим:
  \begin{gather*}
    |a_n(x) b_n(x) + \dots + a_m(x) b_m(x)| \leq 2B \\
    (|a_n(x) + 2a_m(x)| < \varepsilon, \forall x \in \mathbb{E},
    m \geq n \geq N(\varepsilon))
  \end{gather*}
  В силу критерия Коши \eqref{th223}, ряд \eqref{th231:series1} сходится равномерно.
\end{proof}

\begin{theorem}[Признак Абеля]
  \label{th232}
  \begin{gather}
    \sum\limits_{n = 1}^{\infty} a_n(x) b_n(x) \label{th232:series1}
  \end{gather}
  \begin{enumerate}
    \item Если $a_n(x) \in \mathbb{R}, b_n(x) \in \mathbb{C}, n = 1, 2, \dots,
      x \in \mathbb{E}$.
    \item $\{a_n(x)\}$ ограничена на множестве $\mathbb{E}$ и монотонна
      $\forall x \in \mathbb{E}$.
    \item Ряд $\sum b_n(x)$ равномерно сходится на $\mathbb{E}$.
  \end{enumerate}
  Тогда ряд \eqref{th232:series1} равномерно сходится.
\end{theorem}

\begin{proof}
  Доказательство легко провести так, как была доказана теорема \eqref{th161}.
\end{proof}

\begin{example}
  $\sum\limits_{n = 1}^{\infty} \frac{c_n}{n^x}$ ---
  ряд Дирихле. \\
  Если этот ряд сходится в точке $x_0$, то он сходится равномерно
  $\forall x \in \mathbb{E}, \ \mathbb{E} = [x_0, +\infty)$. \\
  Можно воспользоваться Признаком Абеля:
  $$a_n(x) = \frac{1}{n^{x-x_0}}, \ b_n = \frac{c_n}{n^{x_0}}$$
\end{example}

\begin{exercise}
  Рассмотреть и доказать абсолютную сходимость при $x > x_0 + 1$
\end{exercise}

\section{Равномерная сходимость и непрерывность}

\begin{theorem}
  Пусть $f_n \stackrel{\mathrm{\mathbb{E}}}{\rightrightarrows} f, \ x_0$ ---
  предельная точка множества $\mathbb{E}$ и пусть
  $\lim\limits_{x \to x_0} f_n(x) = A_n, \ (n = 1, 2, \dots).$
  Тогда $\{A_n\}$ сходится и
  \begin{gather}
    \lim\limits_{x \to x_0} f(x) = \lim\limits_{n \to \infty} A_n \label{th241:lim1}
  \end{gather}
  Иными словами, 2 предельных перехода в данном случае коммутируют. \\
  $$\lim\limits_{x \to x_0} \lim\limits_{n \to \infty} f_n(x) =
  \lim\limits_{n \to \infty} \lim\limits_{x \to x_0} f_n(x)$$
\end{theorem}

\begin{proof}
  Пусть $\varepsilon > 0.$ В силу равномерной сходимости последовательности
  $\{f_n\} \ \exists N : n > N, m > N, x \in \mathbb{E},$
  \begin{gather}
    \left|f_n(x) - f_m(x)\right| < \varepsilon \label{th241:uneq1}
  \end{gather}
  Переходя в неравенстве \eqref{th241:uneq1} к приделу при $x \to x_0$ получим
  \begin{gather}
    \left|A_n - A_m \right| < \varepsilon, \ (n, m > N) \label{th241:uneq2}
  \end{gather}
  Поэтому $\{A_n\}$ --- последовательность для которой выполняется признак
  Коши сходимости последовательности $\Rightarrow$ она сходится. \\
  Обазначим ее предел $A$ \\
  \begin{gather}
    \left|f(x) - A\right| \leq \left|f(x) - f_n(x)\right| +
    \left|f_n(x) - A_n\right| + \left|A_n - A\right| \label{th241:uneq3}
  \end{gather}
  Выберем $n:$
  \begin{gather}
    |f(x) - f_n(x)| < \frac{\varepsilon}{3}, \ \forall x \in \mathbb{E}
    \label{th241:uneq4}
  \end{gather}
  Это возможно в силу равномерной сходимости.
  \begin{gather}
    |A_n - A| < \frac{\varepsilon}{3} \label{th241:uneq5}
  \end{gather}
  Затем, для этого $n$ подберем такую окрестность
  $U(x_0): x \in U(x_0), x \not = x_0,$ следовательно:
  \begin{gather}
    |f_n(x) - A_n| < \frac{\varepsilon}{3} \label{th241:uneq6}
  \end{gather}
  Из неравенств \eqref{th241:uneq3} --- \eqref{th241:uneq6} получим
  $$|f(x) - A| < \varepsilon, \ \forall x \in U(x_0), \ x \not = x_0$$
  Это равносильно равенству \eqref{th241:lim1}
\end{proof}

\begin{theorem}
  \label{th242}
  Последовательность функций, непрерывных в точке $x \in \mathbb{E}
  f_n \stackrel{\mathrm{\mathbb{E}}}{\rightrightarrows} f,$ то функция
  $f$ непрерывна в точке $x_0$.
\end{theorem}

\begin{proof}
  Без доказательства.
\end{proof}

\begin{remark}
  Обратное не верно, то есть последовательность непрерывных функций может
  неравномерно сходиться.
\end{remark}

Из теоремы \eqref{th242} и определения \eqref{def222} $\Rightarrow$ теорема \eqref{th243}

\begin{theorem}
  \label{th243}
  Если функции $U_n(x), \ (n = 1, 2, \dots), \ x \in \mathbb{E}$ непрерывны
  в точке $x_0 \in \mathbb{E}$ и ряд $\sum\limits_{n = 1}^{\infty} U_n(x)$
  равномерно сходится на $\mathbb{E}$, то его сумма $f(x)$ также непрерывна
  в точкке $x_0$.
\end{theorem}

\section{Равномерная сходимость и интегрирование}
\begin{theorem}
  \label{th251}
  Пусть $f_n$ --- последовательность действительных, значимых, интегрируемых на
  отрезке $[a, b]$ функций. Тогда функция $f$ также интегрируема на $[a,b]$ и
  \begin{gather}
    \int\limits_a^b f(x) dx = \lim\limits_{n \to \infty} \int\limits_a^b f_n(x) dx
    \label{th251:int1}
  \end{gather}
  Существование предела заранее не предполагается.
\end{theorem}

\begin{proof}
  $\forall \varepsilon > 0, \ \exists n:$
  \begin{gather}
    |f_n(x) - f(x)| < \varepsilon, \ x \in [a,b] \label{th251:uneq1}
  \end{gather}
  Зафиксируем $n$ и выберем разбиение
  $[a,b], \ \triangle_1, \dots, \triangle_S$ так, чтобы выполнялось неравенство
  \begin{gather}
    \sum_i \omega(f_n, \triangle_i) |\triangle_i| < \varepsilon \label{th251:sum1}
  \end{gather}
  \begin{comment}
    $\omega(f, E) = \sup - \inf$ --- колебание функции.
  \end{comment}
  Функции $f_n$ интегрируемы на $[a,b]$. По скольку
  $\omega(f, \triangle_i) \leq \omega(f_n, \triangle_i) + 2 \varepsilon,  \ (i = 1, \dots, S)$
  (смотри \eqref{th251:uneq1}).
  $$\sum_i \omega(f, \triangle_i)|\triangle_i| \leq \varepsilon + 2\varepsilon (b - a)$$
  Отсюда следует, что $f \in \mathbb{R} [a,b]$. Для доказательства \eqref{th251:int1}
  выберем $n > N: $
  \begin{gather*}
    |f_n(x) - f(x)| < \varepsilon, \ (a \leq x \leq b), \ n > N \\
    \left|\int\limits_a^b f(x) dx - \int\limits_a^b f_n(x) dx\right| \leq
    \int\limits_a^b |f(x) - f_n(x)| dx < \varepsilon(b - a)
  \end{gather*}
  Отсюда вытекает \eqref{th251:int1}. \\
\end{proof}

\begin{theorem}
  $U_n \in R[a,b]$ (Интегрируема). Если
  \begin{gather}
    f(x) = \sum\limits_{n = 1}^{\infty} U_n(x), \ (a \leq x \leq b)
    \label{th252:series1}
  \end{gather}
  При чем ряд \eqref{th252:series1} сходится на $[a,b]$, тогда
  $$\int\limits_a^b f(x) dx = \sum\limits_{n = 1}^{\infty} \int\limits_a^b
  f(x) dx$$
  Иными словами ряд \eqref{th252:series1} можно интегрировать частями.
\end{theorem}

\begin{proof}
  Без доказательства.
\end{proof}

\begin{remark}
  При нарушении равномерности ряд, состоящий из интегрируемых функций может
  иметь интегрируемую сумму.
\end{remark}

\section{Равномерная сходимость и дифференцирование}
$f_n(x) = \frac{\sin nx}{\sqrt{n}}, x \in \mathbb{R}$ показывает, что из
равномерной сходимости последовательности функций не следует даже поточечная
сходимость последовательностей функций производных. \\
То есть нужны более сильные предположения, чтобы заключать, что $f_n' \to f_n$,
при $f_n \to f$.

\begin{theorem}
  \label{th261}
  Пусть $f_n(x) \to f(x), \ x \in [a,b], \ n \to \infty, \ f_n \in C[a,b], \
  (n = 1, 2, \dots)$. \\
  Если $\{f_n'(x)\}$ сходится равномерно на $[a,b]$, то $f_n(x)$ дифференцируема
  и $$f'(x) = \lim\limits_{n \to \infty} f_n'(x)$$
\end{theorem}

\begin{proof}
  Обозначим через $f^*$ предел последовательности $f_n'$. Ввиду теоремы \eqref{th242}
  $f^*$ непрерывна на $[a,b]$. \\
  Применим теорему \eqref{th251} к последовательости $\{f_n\}$ на промежутке
  $[a, x],$ где $x \in [a,b]$
  $$\int\limits_a^x f^*(t) dt = \lim \int\limits_a^x f'(t) dt =
  \lim\limits_{n \to \infty} (f_n(x) - f_n(a)) = f(x) - f(a)$$
  Так как интеграл слева ввиду непрерывности функции $f^*$ имеет производную
  равную $f',$ то ту же производную имеет и $f(x)$.
  $$f'(x) = f^*(x) = \lim\limits_{n \to \infty} f'(x), x \in [a,b]$$
\end{proof}

Перефразируем теорему \eqref{th261} с точки зрения рядов: \\
Пусть сходящийся ряд $\sum\limits_{n = 1}^{\infty} U_n(x) =: f(x), x \in [a,b]$
и пусть $U_n(x) \in C^1[a,b], \ (n = 1, 2, \dots)$.\\
Если ряд $\sum\limits_{n = 1}^{\infty} U_n'(x)$ сходится равномерно на $[a,b]$,
то сумма $f(x)$ дифференцируема, и \\
$f'(x) = \sum\limits_{n = 1}^{\infty} U_n'(x), x \in [a,b]$.

