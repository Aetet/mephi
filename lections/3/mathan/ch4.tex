\section{Ортогональные системы} \label{ch41}
В параграфах \eqref{ch41} - \eqref{ch43} $\mathbb{X}$ --- линейное
бесконечномерное пространство(действительное или комплексное, со скалярным
произведением).
$$X(\cdot \ , \ \cdot), \ \|x\| = \sqrt{(x, x)}.$$
$\mathbb{K}$ --- некоторое счетное или конечное множество.
\begin{definition}
  \label{def411}
  Система векторов $\{x_k: k \in \mathbb{K}\}, \ x \in \mathbb{X}$ ---
  ортогональная система(ОС). \\
  $(x_i, x_j) = 0, \ \forall i, j \in \mathbb{K}, i \not = j$ (и система не
  нулевая). Если $(x_i, x_i) = 1$, то система называется ортонормированной.
\end{definition}

\begin{theorem}
  Ортогональная система векторов линейно независима, то есть линейно не
  зависима каждая ее конечная подсистема.
\end{theorem}
\begin{proof}
  Определение линейной независимости:
  \begin{gather*}
    \alpha_1 x_1 + \dots + \alpha_i x_i +
    \dots = 0 \Longleftrightarrow \alpha_i = 0, \ \forall i
  \end{gather*}
  Скалярно умножим все члены на $x_i$, тогда получим:
  \begin{gather*}
    \label{th411:eq1}
    \alpha_1 (x_1, x_i) + \dots + \alpha_i (x_i, x_i) + \dots = (0, x_i) \\
    \label{th411:eq2}
    \alpha_i (x_i, x_i) = 0 \\
    \label{th411:eq3}
    \alpha_i = 0
  \end{gather*}
  Равенство \eqref{th411:eq2} следует из определения \eqref{def411}, равенство
  \eqref{th411:eq3} следует из того, что $(x_i, x_i)~\not=~0$ (так как система
  не нулевая).
\end{proof}

\section{Коэффициенты Фурье}
\begin{definition}
  Пусть $\{e_k: \ k \in \mathbb{K}\}$ --- ОНС в $\mathbb{X}$, $\{(x, e_k)\}, x
  \in \mathbb{X}$ называется коэффициентами Фурье элемента $x$ в ОНС $e_k$.
\end{definition}
\begin{lemma}
  \label{lemma421}
  Если система векторов $e_1, \dots, e_n$ пространства $\mathbb{X}$ --- ОН, то
  $\forall x \in \mathbb{X}$ вектор $h = x - x_e$, где
  \begin{gather}
    x_e = \sum\limits_{k = 1}^{n} (x, e_k) e_k
    \label{lemma1:eq1}
  \end{gather}
  ортоганален подпространству $\mathbb{L} = \langle e_1, \dots, e_n \rangle$ (натянотому
  на векторы $e_1, \dots, e_n$)
\end{lemma}

\begin{proof}
  Достаточно проверить, что скалярное произведение \\ $(h, e_j) = 0, \ \forall j =
  1, \dots, n$
  \begin{gather*}
    (h, e_j) = (x, e_j) - \sum\limits_{k = 1}^{n} (x, e_k) (e_k, e_j) =
    (x, e_j) - (x, e_j) = 0
  \end{gather*}
\end{proof}

\begin{lemma}[теорема Пифагора]
  \label{lemma422}
  Если векторы $x_1, \dots, x_n$ попарно ортогональны и $x = x_1 + \dots +
  x_n$, то $\|x\|^2 = \|x_1\|^2 + \dots + \|x_n\|^2$
\end{lemma}

\begin{proof}
  $(x, x) = (\sum\limits_{i = 1}^{n} x_i, \sum\limits_{i = 1}^{n} x_i) =
  \sum\limits_{i, j = 1}^{n} (x_i, x_j) = \sum\limits_{i = 1}^{n} (x_i, x_i)$
\end{proof}

\begin{theorem}[экстремальное свойство коэффициентов Фурье]
  \label{th423}
  Если $e_1, \dots, e_n$ --- ОНС пространства $\mathbb{X}$, то $\forall x \in
  \mathbb{X}$ и $\forall y = \alpha_1 e_1 + \dots + \alpha_n e_n$ имеет место
  неравенство:
  \begin{gather*}
    \|x - \sum\limits_{k = 1}^{n} (x, e_k) e_k \| \leq \|x - \sum\limits_{k =
    1}^{n} \alpha_k e_k \|,
  \end{gather*}
  в котором равенство возможно при условии: $\alpha_k = (x, e_k) \ \forall k = 1,
  \dots, n$.
\end{theorem}

\begin{proof}
  Представим $x - y$ в виде $x - y = (x_e - y) + h$, где $x_e, h$ определены в
  лемме~\eqref{lemma421}. \\
  По лемме~\eqref{lemma421} $h \perp (x_e - y) \in \mathbb{L}$. По теореме
  Пифагора (лемма~\ref{lemma422}):
  \begin{gather*}
    \|x - y\|^2 = \|x_e - y\|^2 + \|h\|^2 = \|x_e - y\|^2 + \|x - x_e\|^2 \geq
    \|x - x_e\|^2
  \end{gather*}
  равенство возможно, когда коэффициенты $\alpha_k$ совпадают с коэффициентами
  Фурье.
\end{proof}

\begin{remark}
  Теорема \eqref{th423} показывает, что вектор $x_e$ является наилучшей в
  смысле нормы пространства $\mathbb{X}$, аппроксимацией вектора $x$
  подпространства $\mathbb{L} = \langle e_1, \dots, e_n \rangle$, так что
  наименьшее уклонением вектора $x$ от $\mathbb{L}$ равно $\|x - x_e\|$.
\end{remark}

\section{секция}
\label{ch43}
