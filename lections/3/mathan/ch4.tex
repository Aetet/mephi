\section{Ортогональные системы} \label{ch41}
В параграфах \eqref{ch41} - \eqref{ch43} $\mathbb{X}$ --- линейное
бесконечномерное пространство(действительное или комплексное, со скалярным
произведением).
$$X(\cdot \ , \ \cdot), \ \|x\| = \sqrt{(x, x)}.$$
$\mathbb{K}$ --- некоторое счетное или конечное множество.
\begin{definition}
  \label{def411}
  Система векторов $\{x_k: k \in \mathbb{K}\}, \ x \in \mathbb{X}$ ---
  ортогональная система(ОС). \\
  $(x_i, x_j) = 0, \ \forall i, j \in \mathbb{K}, i \not = j$ (и система не
  нулевая). Если $(x_i, x_i) = 1$, то система называется ортонормированной.
\end{definition}

\begin{theorem}
  Ортогональная система векторов линейно независима, то есть линейно не
  зависима каждая ее конечная подсистема.
\end{theorem}
\begin{proof}
  Определение линейной независимости:
  \begin{gather*}
    \alpha_1 x_1 + \dots + \alpha_i x_i +
    \dots = 0 \Longleftrightarrow \alpha_i = 0, \ \forall i
  \end{gather*}
  Скалярно умножим все члены на $x_i$, тогда получим:
  \begin{gather*}
    \label{th411:eq1}
    \alpha_1 (x_1, x_i) + \dots + \alpha_i (x_i, x_i) + \dots = (0, x_i) \\
    \label{th411:eq2}
    \alpha_i (x_i, x_i) = 0 \\
    \label{th411:eq3}
    \alpha_i = 0
  \end{gather*}
  Равенство \eqref{th411:eq2} следует из определения \eqref{def411}, равенство
  \eqref{th411:eq3} следует из того, что $(x_i, x_i)~\not=~0$ (так как система
  не нулевая).
\end{proof}

\section{Коэффициенты Фурье}
\begin{definition}
  Пусть $\{e_k: \ k \in \mathbb{K}\}$ --- ОНС в $\mathbb{X}$, $\{(x, e_k)\}, x
  \in \mathbb{X}$ называется коэффициентами Фурье элемента $x$ в ОНС $e_k$.
\end{definition}
\begin{lemma}
  \label{lemma421}
  Если система векторов $e_1, \dots, e_n$ пространства $\mathbb{X}$ --- ОН, то
  $\forall x \in \mathbb{X}$ вектор $h = x - x_e$, где
  \begin{gather}
    x_e = \sum\limits_{k = 1}^{n} (x, e_k) e_k
    \label{lemma1:eq1}
  \end{gather}
  ортоганален подпространству $\mathbb{L} = \langle e_1, \dots, e_n \rangle$ (натянотому
  на векторы $e_1, \dots, e_n$)
\end{lemma}

\begin{proof}
  Достаточно проверить, что скалярное произведение \\ $(h, e_j) = 0, \ \forall j =
  1, \dots, n$
  \begin{gather*}
    (h, e_j) = (x, e_j) - \sum\limits_{k = 1}^{n} (x, e_k) (e_k, e_j) =
    (x, e_j) - (x, e_j) = 0
  \end{gather*}
\end{proof}

\begin{lemma}[теорема Пифагора]
  \label{lemma422}
  Если векторы $x_1, \dots, x_n$ попарно ортогональны и $x = x_1 + \dots +
  x_n$, то $\|x\|^2 = \|x_1\|^2 + \dots + \|x_n\|^2$
\end{lemma}

\begin{proof}
  $(x, x) = (\sum\limits_{i = 1}^{n} x_i, \sum\limits_{i = 1}^{n} x_i) =
  \sum\limits_{i, j = 1}^{n} (x_i, x_j) = \sum\limits_{i = 1}^{n} (x_i, x_i)$
\end{proof}

\begin{theorem}[экстремальное свойство коэффициентов Фурье]
  \label{th423}
  Если $e_1, \dots, e_n$ --- ОНС пространства $\mathbb{X}$, то $\forall x \in
  \mathbb{X}$ и $\forall y = \alpha_1 e_1 + \dots + \alpha_n e_n$ имеет место
  неравенство:
  \begin{gather*}
    \|x - \sum\limits_{k = 1}^{n} (x, e_k) e_k \| \leq \|x - \sum\limits_{k =
    1}^{n} \alpha_k e_k \|,
  \end{gather*}
  в котором равенство возможно при условии: $\alpha_k = (x, e_k) \ \forall k = 1,
  \dots, n$.
\end{theorem}

\begin{proof}
  Представим $x - y$ в виде $x - y = (x_e - y) + h$, где $x_e, h$ определены в
  лемме~\eqref{lemma421}. \\
  По лемме~\eqref{lemma421} $h \perp (x_e - y) \in \mathbb{L}$. По теореме
  Пифагора (лемма~\ref{lemma422}):
  \begin{gather*}
    \|x - y\|^2 = \|x_e - y\|^2 + \|h\|^2 = \|x_e - y\|^2 + \|x - x_e\|^2 \geq
    \|x - x_e\|^2
  \end{gather*}
  равенство возможно, когда коэффициенты $\alpha_k$ совпадают с коэффициентами
  Фурье.
\end{proof}

\begin{remark}
  Теорема \eqref{th423} показывает, что вектор $x_e$ является наилучшей в
  смысле нормы пространства $\mathbb{X}$, аппроксимацией вектора $x$
  подпространства $\mathbb{L} = \langle e_1, \dots, e_n \rangle$, так что
  наименьшее уклонением вектора $x$ от $\mathbb{L}$ равно $\|x - x_e\|$.
\end{remark}

\begin{theorem}[неравенство Бесселя]
  \label{th424}
  Если $\{e_1, \dots, e_n\}$ --- ОНС в $\mathbb{X}$, то $\forall x \in
  \mathbb{X}$ справедливо неравенство:
  \begin{gather}
    \sum\limits_{k = 1}^{n} |(x, e_k)|^2 \leq \|x\|^2
    \label{th424:uneq1}
  \end{gather}
  Если $\{e_k: k \in \mathbb{K}\}$ --- ОНС, то $\forall x \in \mathbb{X}$
  \begin{gather}
    \sum |(x, e_k)|^2 \leq \|x\|^2
    \label{th424:uneq2}
  \end{gather}
\end{theorem}

\begin{proof}
  По лемме \eqref{lemma421}
  \begin{gather*}
    x = \sum\limits_{k = 1}^{n} (x, e_k)e_k + h,
  \end{gather*}
  при чем система векторов $e_1, \dots, e_n, h$ --- ортогональна в $\mathbb{X}$
  \\
  по теореме Пифагора получаем:
  \begin{gather}
    \|x\|^2 = \sum\limits_{k = 1}^{n} |(x, e_k)|^2 + \|h\|^2
    \label{th424:eq1}
  \end{gather}
  остюда следует \eqref{th424:uneq1}, (так как это имеет место для любой
  конечной системы векторов), отсюда следует \eqref{th424:uneq2}.
\end{proof}

\begin{remark}
  Из \eqref{th424:eq1} следует формула наименьшего отклонения:
  \begin{gather}
    \|x - x_e\|^2 \equiv \|x - \sum\limits_{k = 1}^{n} (x, e_k) e_k \|^2 =
    \|x\|^2 - \sum\limits_{k = 1}^{n} |(x, e_k)|^2
    \label{th424:eq3}
  \end{gather}
\end{remark}

Переформулируем понятие коэффициентов Фурье для произвольной ОС(не обязательно
нормированной) $\{f_k\}$. \\
Для этого по этой системе построим ОНС. \\
$\{e_k = \frac{f_k}{\|f_k\|}\}$, используем ортгональное разложение:
\begin{gather*}
  x = x_e +
  h, \ x = \sum\limits_{k = 1}^{n} (x, e_k) e_k + h = \sum\limits_{k =
  1}^{n} \frac{(x, f_k)}{\|f_k\|^2} f_k + h
\end{gather*}

\begin{definition}
  \label{def422}
  $\{e_k = \frac{(x, f_k)}{\|f_k\|^2} \}$ --- называется коэффициентами Фурье
  вектора $x$ в ОС~$\{f_k\}$. \\
  Заменим в неравенстве \eqref{th424:uneq2}, $e_k$ на $\frac{f_k}{\|f_k\|}$
  получим неравенство Бесселя для произвольной ОС.
  \begin{gather}
    \sum\limits_{k \in \mathbb{K}} \frac{|(x, f_k)|^2}{\|f_k\|^2} \leq \|x\|^2,
    \ \{f_k, k \in \mathbb{K}\}
    \label{def422:uneq1}
  \end{gather}
  Или, в других обозначения:
  \begin{gather*}
    \sum\limits_{k \in \mathbb{K}} |C_k|^2 \|f_k\|^2 \leq \|x\|^2
  \end{gather*}
\end{definition}

\begin{example}
  В пространстве $\mathbb{X} = \mathcal{R}_2([-\pi, \pi], \mathbb{C})$. \\
  Рассмотрим ОС $\{e^{ikt}: k \in \mathbb{Z}\}$. \\
  В соответствии с определением \eqref{def422} коэффициенты Фурье $C_k$ функции
  $f$ в системе $\{e_{ik}\}$ выражаются формулами:
  \begin{gather}
    C_k = \frac{1}{2\pi} \int\limits_{-\pi}^\pi f(t) e^{-ikt} dt
    \label{ex421:coef1}
  \end{gather}
  из неравенства Бесселя \eqref{def422:uneq1} $\forall f \in
  \mathcal{R}_2([-\pi,\pi], \mathbb{C})$
  \begin{gather}
    \sum\limits_{k = -\infty}^{\infty} |C_k|^2 \leq \frac{1}{2\pi}
    \int\limits_{-\pi}^\pi |f(t)|^2 dt
    \label{ex421:uneq1}
  \end{gather}
\end{example}

\begin{example}
  Аналогично находим коэффициенты Фурье. \\
  $\{\frac{1}{2} a_0, a_k, b_k : k \in \mathbb{N}\}$ функции $f \in
  \mathcal{R}([-\pi, \pi], \mathbb{C})$ в ОС $\{1, \cos kx, \sin kx : k \in
  \mathbb{N}\}$:
  \begin{gather}
    a_k = \frac{1}{\pi} \int\limits_{-\pi}^\pi f(t) \cos kt dt, \ k = 0, 1,
    \dots \label{ex422:coef1} \\
    b_k = \frac{1}{\pi} \int\limits_{-\pi}^\pi f(t) \sin kt dt, \ k = 1, 2,
    \dots \label{ex422:coef2}
  \end{gather}
  по неравенству Бесселя все принимает вид:
  \begin{gather}
    \frac{|a_0|^2}{2} + \sum\limits_{k = 1}^{\infty} (|a_k|^2 + |b_k|^2) \leq
    \frac{1}{\pi} \int\limits_{-\pi}^\pi |f(t)|^2 dt \label{ex422:uneq1}
  \end{gather}
\end{example}

\begin{remark}
  Сравнивая равенства \eqref{ex422:coef1}, \eqref{ex422:coef2} и
  \eqref{ex421:coef1} с учетом формулы Эйлера получаем:
  \begin{gather}
    C_k =
    \begin{cases}
      \frac{1}{2} (a_k - ib_k), \ k \geq 0 \\
      \frac{1}{2} (a_k + ib_{-k}), \ k < 0
    \end{cases}
    \label{rm423:cases1}
  \end{gather}
\end{remark}

\section{Ряд Фурье}
\label{ch43}
\begin{definition}
  Если $\{f_1, \dots, f_k, \dots\}$ --- ОС в $\mathbb{X}$, а $x \in
  \mathbb{X}$, то можно составить ряд:
  \begin{gather*}
    x \sim \sum\limits_{k = 1}^{\infty} C_k f_k,
  \end{gather*}
  где $C_k = \frac{(x, f_k)}{\|f_k\|^2}$. \\
  Этот ряд называется рядом Фурье вектора $x$ по ОС $\{f_k\}$. \\
  Ряд Фурье по ОНС $\{e_k\}$ имеет вид:
  \begin{gather*}
    x \sim \sum\limits_{k = 1}^{\infty} (x, e_k) e_k
  \end{gather*}
\end{definition}

\begin{definition}
  Говорят, что ряд $\sum\limits_{k = 1}^{\infty} y_k, y_k \in \mathbb{X}$
  сходится в $\mathbb{X}$ к вектору $x \in \mathbb{X}$ (сходится по
  норме($\|\cdot\|$) пространства $\mathbb{X}$), если
  \begin{gather*}
    \lim\limits_{n \to \infty} \|x - \sum\limits_{k = 1}^{n} y_k\| = 0
  \end{gather*}
  При этом пишем $x \overset{\mathbb{X}}= \sum\limits_{k = 1}^{\infty}
  y_k$ по норме пространства $\mathbb{X}$.
\end{definition}

\begin{theorem}
  \label{th431}
  $\{e_k: k \in \mathbb{N}\}$ --- ОНС в $\mathbb{X}, x \in \mathbb{X}$, где \\
  $x \overset{\mathbb{X}} = \sum\limits_{k = 1}^{\infty} (x, e_k) e_k
  \Longleftrightarrow$ когда $\|x\|^2 = \sum\limits_{k = 1}^{\infty} |(x,
  e_k|^2$. \\
  Это равенство называется равенством Парсеваля и представляет собой обобщение
  теоремы Пифагора на случай бесконечномерного пространства.
\end{theorem}

\begin{definition}
  Система $\{x_k: k \in \mathbb{K} \}$ векторов в пространстве $\mathbb{X}$
  называется полной в множестве $\mathbb{E} \subset \mathbb{X}$, если любой
  вектор $x \in E$ можно сколь угодно точно в смысле нормы пространства
  $\mathbb{X}$ приблизить к конечной линейной комбинации векторов системы.
\end{definition}

\begin{theorem}
  \label{th432}
  Пусть $\{e_1, \dots, e_n, \dots\}$ --- ОНС в $\mathbb{X}$, тогда следующее
  условие эквивалентны:
  \begin{enumerate}
    \item $\{e_k\}$ полна в множестве $\mathbb{E} \subset \mathbb{X}$.
    \item $\forall x \in \mathbb{E} \subset \mathbb{X}$ имеет место разложение
      (в ряд Фурье) $x \overset{\mathbb{X}} = \sum\limits_{k = 1}^{\infty} (x,
      e_k) e_k$
    \item $\forall x \in \mathbb{E} \subset \mathbb{X}$ имеет место равенство
      Парсеваля: $\|x\|^2 = \sum\limits_{k = 1}^{\infty} |(x, e_k)|^2$.
  \end{enumerate}
\end{theorem}

\begin{proof}
  Из $1 \Rightarrow 2$ в силу экстремального свойства коэффициентов Фурье. \\
  Из $2 \Rightarrow 3$ по теореме \eqref{th431} \\
  Из $3 \Rightarrow 1$, по скольку по формуле уклонений
  \begin{gather*}
    \|x - \sum\limits_{k = 1}^{n} (x, e_k)\|^2 = \|x\|^2 - \sum\limits_{k =
    1}^{n} |(x, e_k)|^2 \to 0, \text{при} \ n \to \infty
  \end{gather*}
\end{proof}

\section{Тригонометрический ряд Фурье}
$\mathbb{X} = \mathbb{R}_2([-\pi, \pi], \mathbb{C}), e_k = \{e^{ikx} : k \in
\mathbb{Z}\}, f \in \mathbb{R}([-\pi, \pi], \mathbb{C})$
\begin{gather}
  C_k(f) = C_k = \frac{1}{2} \int\limits_{-\pi}^\pi f(x) e^{-ikx} dx
  \label{ch44:eq1}
\end{gather}
Сопоставим функцию
\begin{gather}
  f(x) \sim \sum\limits_{k = -\infty}^{+\infty} C_k e^{ikx}
  \label{ch44:eq2}
\end{gather}

\begin{definition}
  Если нам дан тригонометрический ряд Фурье в комплексной записи, то его
  $n$-ая частная сумма равна:
  \begin{gather}
    S_n(x) = S_n(f, x) = \sum\limits_{k = -n}^{n} C_k e^{ikx}
    \label{def441:eq1}
  \end{gather}
\end{definition}

\begin{definition}
  Ряд Фурье функции $f \in \mathbb{R}([-\pi, \pi], \mathbb{C})$ по системе
  $\{1, \cos kx, \sin kx: k \in \mathbb{N}\}$ называется тригонометрическим
  рядом Фурье и записывается следующим образом:
  \begin{gather}
    f(x) \sim \frac{a_0}{2} + \sum\limits_{k = 1}^{\infty} (a_k \cos kx + b_k
    \sin kx)
    \label{def442:sim1} \\
    a_k = \frac{1}{\pi} \int\limits_{-\pi}^\pi f(x) \cos kx dx, \ k = 0, 1,
    \dots
    \label{def442:coef1} \\
    b_k = \frac{1}{\pi} \int\limits_{-\pi}^\pi f(x) \sin kx dx, \ k = 1, 2,
    \dots
    \label{def442:coef2}
  \end{gather}
  Если функция $f$ --- действительная, то $a_k, b_k \in \mathbb{R}$ и
  $\underline{C_k} = \overline{C_k} \ (k = 0, 1, \dots)$.
\end{definition}

\begin{definition}
  Тригонометрические многочлены $D_n(x) = \sum\limits_{k = -n}^{n} e^{ikx},
  K_n(x) = \frac{1}{n + 1} \sum\limits_{m = 0}^{n} D_m(x)$ называются
  соответственно ядром Дирихле и ядром Эйлера.
  \begin{gather*}
    a_1, \dots, a_n \to a, \ \text{при} \ n \to \infty \\
    \frac{a_1 + \dots + a_n}{n} \to a, \ \text{при} \ n \to \infty
  \end{gather*}
  а обратное не верно.
\end{definition}

\begin{gather}
  D_n(x) = \sum\limits_{k = -n}^{n} e^{ikx}
  \label{ch44:kernels} \\
  K_n(x) = \frac{1}{n + 1} \sum\limits_{m = 0}^{n} D_m(x)
\end{gather}

\begin{theorem}
  При $n = 0, 1, \dots$ имеем:
  \begin{gather}
    D_n(x) = \frac{\sin((n + \frac{1}{2})x)}{\sin \frac{x}{2}}
    \label{th441:eq1} \\
    K_n(x) = \frac{1}{n + 1} \frac{1 - \cos ((n + 1)x)}{1 - \cos x}
    \label{th441:eq2} \\
    \frac{1}{\pi} \int\limits_{-\pi}^\pi D_n(x) dx = \frac{1}{2\pi}
    \int\limits_{-\pi}^\pi K_n(x) dx = 1
    \label{th441:eq3}
  \end{gather}
  кроме того,
  \begin{gather}
    K_n(x) \geq 0, K_n(x) \leq \frac{2}{(n+1)(1 - \cos \delta)},
    \text{где} \ 0 < \delta \leq |x| \leq \pi
    \label{th441:uneq1}
  \end{gather}
\end{theorem}

\begin{proof}
  Согласно \eqref{ch44:kernels}
  \begin{gather}
    D_n(x) = \frac{e^{i(n+1)x} - e^{-inx}}{e^{ix} - 1}
    \label{th441:eq4}
  \end{gather}
  чтобы получить \eqref{th441:eq1} домножим здесь числитель и знаменатель на
  $e^{-ix/2}$. \\
  Подставим \eqref{th441:eq4}, в определение ядра $K_n(x)$, получим:
  \begin{gather*}
    K_n(x) = \frac{1}{n + 1} \cdot \frac{e^{-ix} - 1}{(e^{-ix} - 1)(e^{ix} - 1)}
    \sum\limits_{m = 0}^{n} (e^{i(m+1)x} - e^{-imx}) = \\ \frac{1}{n + 1} \cdot
    \frac{1}{2 - (e^{ix} + e^{-ix})} \sum\limits_{m = 0}^{n} [(e^{imx} +
    e^{-imx}) - (e^{i(m+1)x} + e^{-i(m+1)x})] = \\ \frac{1}{n + 1}
    \cdot \frac{1}{2 - 2
    \cos x} (2 - (e^{i(m+1)x} + e^{-i(m+1)x}))
  \end{gather*}
  откуда следует \eqref{th441:eq2}. \\
  Значит $K_n(x) \geq 0$ и выполняется \eqref{th441:uneq1}. А \eqref{th441:eq3}
  непосредственно следует из \eqref{ch44:kernels}.
\end{proof}

Далее предпологаем, что функция $f$, изначально определенная на $[-\pi, \pi]$,
продолжена на $\mathbb{R}$ как $2\pi$-периодическая функция. \\
Если $f \in C[-\pi, \pi]$, то ее $2\pi$-периодическое продолжение непрерывно на
$\mathbb{R}$

\begin{gather*}
  (f \in C_{2\pi}) \Longleftrightarrow f(-\pi) = f(\pi)
\end{gather*}

\begin{lemma}[интегральное представление частичной суммы ряда Фурье]
  \begin{gather}
    \forall x \in \mathbb{R}: \ S_n(f, x) = \frac{1}{2\pi}
    \int\limits_{-\pi}^\pi f(x - t) D_n(t) dt
    \label{lem441:eq1}
  \end{gather}
\end{lemma}

\begin{proof}
  Пусть $S_n$ --- частичная сумма, тогда:
  \begin{align*}
    S_n(f, x) = \sum\limits_{k = -n}^{n} C_k e^{ikx} &= \sum\limits_{k = -n}^{n}
    \frac{1}{2\pi} \int\limits_{-\pi}^\pi f(u) e^{-iku} du \cdot e^{ikx} = \\
    \frac{1}{2\pi} \int\limits_{-\pi}^\pi f(u) \sum\limits_{k = -n}^{n} e^{ik(x
    - u)} du &=  \frac{1}{2\pi} \int\limits_{-\pi}^\pi f(u) D_n(x - u) du = \\
    \frac{1}{2\pi} \int\limits_{x -\pi}^{x + \pi} f(x - t) D_n(t) dt &=
    \frac{1}{2\pi} \int\limits_{-\pi}^\pi f(x - t) D_n(t) dt
  \end{align*}
  \begin{comment}
    Использовалась замена: $x - u = t$.
  \end{comment}
  Отметим, что последнее равенство выполняется, поскольку в следствии
  периодичности функции безразлично по какому инетрвалу интегрировать, лишь бы
  его длина была равна~$2\pi$.
\end{proof}

\begin{definition}
  Средние арифмитические частичных сумм:
  \begin{gather}
    S_n(f, x), \delta_n(f, x) = \delta_n(x) = \frac{S_0(x) + \dots +
    S_n(x)}{n+1}
    \label{def444:eq1}
  \end{gather}
  Называются полиномами Фейера.
\end{definition}

\begin{theorem}[теорема Фейера]
  Если функция $f \in C_{2\pi}$, то
  \begin{gather}
    \delta_n(x) \stackrel{\mathrm{\mathbb{R}}}{\rightrightarrows} f(x)
    \label{th442:fnef1}
  \end{gather}
\end{theorem}

\begin{proof}
  Согласно формулам \eqref{def444:eq1}, \eqref{lem441:eq1} и
  \eqref{ch44:kernels} имеем: \\
  \begin{gather}
    \delta_n(x) = \frac{1}{2\pi} \int\limits_{-\pi}^\pi f(x - t) K_n(t) dt
  \end{gather}
  поэтому из \eqref{th441:eq3} следует, что:
  \begin{gather}
    f(x) - \delta_n(x) = \frac{1}{2\pi} \int\limits_{-\pi}^\pi (f(x) - f(x -
    t)) K_n(t) dt
    \label{th442:eq1}
  \end{gather}
  $\varepsilon > 0, \ M = \max |f(x)|, \ x \in \mathbb{R}$. Поскольку функция
  $f$ --- равномерно непрерывна, то найдется такое $\delta > 0$, что:
  \begin{gather}
    |x - y| < \delta \Rightarrow |f(x) - f(y)| < \frac{\varepsilon}{2}
    \label{th442:cons1}
  \end{gather}
  Согласно \eqref{th441:uneq1} можно затем выбрать такое $N = N(\varepsilon,
  \delta)$, что
  \begin{gather}
    n > N, \delta \leq |t| \leq \pi \Rightarrow K_n(t) \leq
    \frac{\varepsilon}{4M}
    \label{th442:uneq1}
  \end{gather}
  Из \eqref{th442:cons1} и $K_n(t) \geq 0$ получаем:
  \begin{gather}
    \int\limits_{-\delta}^\delta |f(x) - f(x - t)| |K_n(t)| dt <
    \frac{\varepsilon}{2} \int\limits_{-\pi}^\pi K_n(t) dt = \pi \varepsilon, \
    n = 1, 2, \dots
    \label{th442:uneq2} \\
    \left\{\int\limits_{-\pi}^\delta + \int\limits_{\delta}^\pi \right\} =
    |f(x) - f(x-t)| |K_n(t)| dt \leq \frac{\varepsilon}{4M}
    \int\limits_{-\pi}^\pi 2M dt = \pi \varepsilon, \ \forall n > N
    \label{th442:uneq3}
  \end{gather}
  В силу \eqref{th442:cons1}, \eqref{th442:uneq2} и \eqref{th442:uneq3}
  получаем:
  \begin{gather}
    |f(x) - \delta_n(x)| < \varepsilon, \ \forall x \in \mathbb{R}, \ \forall n
    > N
    \label{th442:uneq4}
  \end{gather}
\end{proof}

\begin{consequence}
  Если две непрерывные $2\pi$-периодические функции $f, g$ имеют один и тот же
  ряд Фурье, то $f(x) = g(x), \ \forall x \in \mathbb{R}$
\end{consequence}

\begin{proof}
  Действительно, если $\delta_n(x)$ --- среднее арифметическое этого ряда, то:
  \begin{gather*}
    \delta_n(x) \to f(x), \ \delta_n \to g(x)
  \end{gather*}
\end{proof}

\begin{consequence}
  Если $f \in C_{2\pi}$ и $\int\limits_{-\pi}^\pi f(x) e^{-inx} dx \equiv 0$,
  то $\forall n \in \mathbb{Z}: \ f(x) \equiv 0$. Таким образом ОС $\{e^{ikx}:
  \ k \in \mathbb{Z}\}$ нельзя дополнить ненулевым элементом.
\end{consequence}

\begin{proof}
  Это вытекает из предыдущего следствия, если положить $g = 0$.
\end{proof}

\begin{consequence}
  Ряд Фурье функции $f \in C_{2\pi}$ либо сходится в каждой точке $x$ к функции
  $f(x)$, либо вовсе расходится в этой точке.
\end{consequence}

\begin{proof}
  НАПИСАТЬ ДОКАЗАТЕЛЬСТВО.
\end{proof}

\begin{remark}
  Ряд Фурье для выражения функции в самом деле может в некоторых точках
  расходится.
\end{remark}

\begin{theorem}[теорема Вейерштрасса]
  \label{th443}
  Если $f \in C_{2\pi}$, то $\forall \varepsilon > 0$ существует такой
  тригонометрический многочлен $T(x): \ \forall x \in \mathbb{R}$:
  \begin{gather*}
    |f(x) - T(x)| < \varepsilon
  \end{gather*}
\end{theorem}

\begin{proof}
  Без доказательства.
\end{proof}

Из теоремы \eqref{th443} следует теорема \eqref{th444}.
\begin{theorem}[теорема Вейерштрасса]
  \label{th444}
  Если $f \in C[a, b]$, то $\forall \varepsilon > 0$ существует такой
  алгеброический многочлен $P(x): \ \forall x \in [a, b]$:
  \begin{gather*}
    |f(x) - P(x)| < \varepsilon
  \end{gather*}
\end{theorem}

\begin{proof}
  Положив $t = \frac{x - a}{b - a} \pi,$ и $x = \frac{b - a}{\pi}t + a$, получим
  функцию $\varphi(t) = f(a + \frac{b - a}{\pi}t)$ на отрезке $[0, \pi]$.
  Продолжим ее в начале четным образом $\varphi(-t) = \varphi(t), \ t \in [\pi,
  0)$. Найдем по теореме \eqref{th443} такой тригонометрический полином
  $T(x): |\varphi(t) - T(t)| < \frac{\varepsilon}{2}, \ \forall t \in
  \mathbb{R}$. Всякий тригонометрический полином раскладывается по Тейлору,
  сходится равномерно на любом конечном интервале. \\
  Пусть $P_n$ --- частичная сумма ряда Тейлора для $T(t)$ такая что: $|T(t) -
  P_n(t)| < \frac{\varepsilon}{2}, \ 0 \leq t \leq \pi$. Тогда $|\varphi_n(t) -
  P_n(t)| < \varepsilon$, при $0 \leq t \leq \pi$. Сделав обратную замену в
  $P_n(t): t = \frac{x - a}{b - a} \pi$, получим многочлен $Q_n(x)$,
  удовлетворяющий условию: $|f(x) - Q_n(x)| < \varepsilon, \ a \leq x \leq b$.
\end{proof}

Получаем еще одно следствие от теоремы Фейера --- полнота тригонометрической
системы функций $C_2[-\pi, \pi]$ и более общо $\mathcal{R}_2[-\pi, \pi]$.

\begin{theorem}[о полноте тригонометрической системы]
  \label{th445}
  Любая функция $f$ из множества $f \in \mathcal{R}[-\pi, \pi]$ может быть сколь угодно
  точно приближена в среднем, то есть по норме:
  \begin{enumerate}
    \item Кусочно-постоянной функции $[-\pi, \pi]$
    \item Непрерывными на отрезке $[-\pi, \pi]$ функциями, принимающие равные
      значения на концах $[-\pi, \pi]$
    \item Тригонометрическими полиномами
  \end{enumerate}
\end{theorem}

\begin{proof}
  Достаточно рассмотреть случай действительно значимых функций.
  \begin{enumerate}
    \item Поскольку $f$ --- интегрируема, то: $\forall \varepsilon > 0 \
      \exists$ разбиение $-\pi = x_0 < x_1 < \dots < x_n = \pi$ отрезка $[-\pi,
      \pi]$, что: $0 \leq \triangle := \int\limits_{-\pi}^\pi f(x) dx -
      \sum\limits_{i = 1}^{n} m_i \triangle x_i < \varepsilon$, где
      \begin{gather*}
        m_i = \inf\{f(x)\}, \ x \in [x_{i-1}, x_i), \ \triangle x_i = x_i - x_{i-1}
      \end{gather*}
      Полагая
        $g(x) =
        \begin{cases}
          m_i, \text{если} \ x \in [x_{i-1}, x_i) \\
          0, \text{если} \ x = \pi \\
        \end{cases} $,
      $M_f = \sup\{|f(x)|\}, |x| \leq \pi$. \\
      Получим:
        $\int\limits_{-\pi}^\pi (f(x) - g(x))^2 dx \leq \int\limits_{-\pi}^\pi
        (|f(x)| + |g(x)|) (|f(x)| - |g(x)|) dx \leq \\
        2M_f \int\limits_{-\pi}^\pi
        (f(x) - g(x)) dx = 2M_f\triangle \leq 2M_f \varepsilon$.
      \item Достаточно уметь приблежать к среднему кусочно постоянной функции.
        Пусть $g$ --- такая функция, с точками разрыва $x_1, \dots, x_n$.
        Удобно присваивать $-\pi = x_1, x_n = \pi$. \\
        Очевидно, какого бы ни было $\varepsilon > 0 \ \exists \delta > 0$, что
        $\delta$-окрестности точек $x_1, \dots, x_n$ не пересекаются и
        $2\delta_n M < \varepsilon$, где $M = \sup \{|g(x)| : |x| \leq \pi\}$
        Заменим функцию $g$ на каждом из отрезков $[-\pi, -\pi + \delta], [x_1
        -\delta, x_1 + \delta], (i = 2, \dots, n-1), [\pi - \delta, \pi]$
        линейной функцией, принимающей на концах этих отрезков соответственно: \\
        $0, g(-\pi + \delta), g(x_i - \delta), g(x_i + \delta), (i = 2, \dots,
        n - 1), g(\pi - \delta), 0$. Получим кусоно линейную, непрерывную на
        кусочном отрезке $[-\pi, \pi]$ функцию $h, h(-\pi) = h(\pi) = 0, \\
        |h(x)| \leq M, \ \forall x \in [-\pi, \pi]$. \\
        Значит
        \begin{gather}
          \int\limits_{-\pi}^\pi (g - h)^2 dx \leq 2M
          \int\limits_{-\pi}^\pi (|g - h|) dx = \\ 2M \sum\limits_{i = 1}^{n}
          \int\limits_{x_i - \delta}^{x_i + \delta} (|g - h|)dx \leq 2M (2M -
          2\delta)n < 4M\varepsilon
        \end{gather}
      \item Осталось показа, что можно приблизить любую функцию класса 2. Но по
        теореме Фейера для любой функции типа $h$, найдется такой
        тригонометрический многочлено, что:
        \begin{gather*}
          \forall \varepsilon > 0, \ T: |h(x) - T(x)| < \varepsilon, \ \forall
          x \in [-\pi, \pi] \\
          \left(\frac{1}{2\pi} \int\limits_{-\pi}^\pi (h(x) - T(x))^2 dx
          \right)^{\frac{1}{2}} < \varepsilon
        \end{gather*}
        Ссылаясь на неравенство треугольника, пространства $\mathcal{R_2}[-\pi, \pi]$
        заключаем, что теорема доказана.
  \end{enumerate}
\end{proof}

Из полноты тригонометрической системы, из теоремы \eqref{th432} (третьего
условия полноты ОС) и форумлы наименьших уклонений следует теорема
\eqref{th446}.

\begin{theorem}
  \label{th446}
  $f \in \mathcal{R}_2([-\pi, \pi], \mathbb{C})$, имеем:
  \begin{enumerate}
    \item $f(x) \overset{\mathcal{R}_2}=\frac{a_0(f)}{2} + \sum\limits_{k = 1}^{\infty}
      a_k(f) \cos kx + b_k(f) \sin kx$, или в комплексной записи: \\
      $f(x) \overset{\mathcal{R}_2}= \sum\limits_{-\infty}^{+\infty} C_k(f) e^{ikx}$, где
      сходимость понимается, как сходимость по норме.
    \item $\frac{1}{\pi} \int\limits_{-\pi}^\pi |f(x)|^2 dx =
      |\frac{a_0(f)}{2}|^2 + \sum\limits_{k = 1}^{\infty} |a_k(f)|^2 +
      |b_k(f)|^2$, или в комплексной записи:\\
      $\frac{1}{2\pi} \int\limits_{-\pi}^\pi |f(x)|^2 dx =
      \sum\limits_{-\infty}^{+\infty} |C_k(f)|^2$.
    \item $\frac{1}{\pi}\int\limits_{-\pi}^\pi |f(x) - S_n(f, x)|^2 dx =
      \sum\limits_{k = n + 1}^{\infty} |a_k(f)|^2 + |b_k(f)|^2 = 4
      \sum\limits_{k = n + 1}^{\infty} |C_k(f)|^2, f \in \mathcal{R}([-\pi, \pi],
      \mathbb{R})$.
  \end{enumerate}
\end{theorem}

\section{Обобщение на неограниченные функции}
\begin{definition}
  Пусть $0 < p < \infty, -\infty \leq a < b \leq + \infty$. Будем писать $f \in
  \mathcal{R}^p[a,b]$, если существует конечное число точек $x_j, j = 0, 1, 2, \dots, n$,
  таких что:
  \begin{enumerate}
    \item $a = x_0 < x_1 < \dots < x_n = b$
    \item Функция $f$ интегрируема по Римману на любом отрезке $f \in
      \mathcal{R}[\alpha, \beta], \ [\alpha, \beta] \subset (x_{j-1}, x_j)$
    \item Интеграл $\int\limits_{x_{j-1}}^{x_j} |f(x)|^p dx, \ j = 1, 2,
      \dots, n$ сходится.
  \end{enumerate}
\end{definition}

\begin{remark}
  Формулы \eqref{ch44:eq1}, \eqref{def442:coef1} и \eqref{def442:coef2},
  определяющие коэффициенты Фурье $C_k(f), a_k(f), b_k(f)$ имеют смысл для
  $\forall f \in \mathcal{R}^1[-\pi, \pi]$, так как тогда $f(x) e^{-ikx}, f(x) \cos kx,
  f(x) \sin kx \in \mathcal{R}^1[-\pi, \pi]$.
\end{remark}

\begin{definition}[неравенство Гельдера]
  \begin{gather*}
    \int\limits_a^b |f(x) g(x)| dx \leq \left(\int\limits_a^b |f(x)|^p dx \right)
    ^{\frac{1}{p}}
    \left(\int\limits_a^b |g(x)|^q dx\right)^{\frac{1}{q}}
  \end{gather*}
  где $q > 1, \ p > 1, \ \frac{1}{p} + \frac{1}{q} = 1$
  \begin{gather*}
    \int\limits_a^b |f(x)| dx \leq (b - a)^{1 - \frac{1}{p}}
    \left(\int\limits_a^b |f(x)|^p dx \right)^{\frac{1}{p}} \\
    \left(\frac{1}{b-a} \int\limits_a^b |f(x)|^{r} dx \right)
    ^{\frac{1}{r}} \leq \left(\frac{1}{b-a} \int\limits_a^b |f(x)|^{h} dx
    \right) ^{\frac{1}{h}}
  \end{gather*}
  если $h > 1, \ 0 < r < h < \infty$.
\end{definition}

\begin{approval}
  теоремы \eqref{th445} и \eqref{th446} остаются в силе, если в них
  пространство $\mathcal{R}_2[-\pi, \pi]$ расширить до линейного пространства
  $\mathcal{R}^2[-\pi, \pi]$:
  \begin{gather*}
    (f, g) = \int f(x) \overline{g(x)} dx
  \end{gather*}
\end{approval}

\begin{proof}
  Смотри теорему \eqref{th445}.
\end{proof}

\begin{lemma}
  \label{lem451}
  Если $f \in \mathcal{R}^p[a,b] (p > 0)$, то $\forall \varepsilon > 0 \exists g \in
  \mathcal{R}[a,b]: \int\limits_a^b |f(x) - g(x)|^p dx < \varepsilon$.
\end{lemma}

\begin{proof}
  Доказать самостоятельно.
\end{proof}

\section{Достаточные условия сходимости тригонометрического ряда Фурье в точке}
\begin{lemma}[Римана]
  Если $f \in \mathcal{R}^1[a,b]$, то
  \begin{gather}
    \int\limits_a^b f(x) e^{i\lambda x} dx \to 0
    \label{lem461:lim1} \\
    \int\limits_a^b f(x) \cos \lambda x dx \to 0
    \label{lem461:lim2} \\
    \int\limits_a^b f(x) \sin \lambda x dx \to 0, \text{при} \ \lambda \to
    \infty, \ \lambda \in \mathbb{R}
    \label{lem461:lim3}
  \end{gather}
\end{lemma}

\begin{proof}
  Будем считать, что функция $f(x)$ --- действительная, так как в случае $f(x)$
  --- комплексная легко сводится к этому, согласно лемме \eqref{lem451} при $p
  = 1$, $\forall \varepsilon > 0, \ \exists g$ (кусочно-постоянная):
  \begin{gather}
    \int\limits_a^b |f(x) - g(x)| dx < \frac{\varepsilon}{2}
    \label{th461:uneq1}
  \end{gather}
  Пусть $g(x) = m, x \in [x_{j-1}, x_j], j = 1, \dots, n$. $(x_0 = a, x_n =
  b)$, тогда:
  \begin{gather*}
    \int\limits_a^b g(x) e^{i\lambda x} dx = \sum\limits_{j = 1}^{n}
    \int\limits_{x_{j-1}}^{x_j} m_j e^{i\lambda x} dx = \left. \frac{1}{ix}
    \sum\limits_{j = 1}^{n} (m_j e^{i \lambda x})
    \right |_{x_{j-1}}^{x_j} \to 0, \ \text{при} \ \lambda \to \infty
  \end{gather*}
  Отсюда из \eqref{th461:uneq1} получим: \\
  \begin{gather*}
    \left| \int\limits_a^b f(x) e^{i\lambda x} dx \right| =
    \left| \int\limits_a^b (f(x) - g(x)) e^{i\lambda x} dx +
    \int\limits_a^b g(x) e^{i\lambda x} dx \right| \leq \\
    \int\limits_a^b |f(x) - g(x)| e^{i\lambda x} dx +
    \left| \int\limits_a^b g(x) e^{i\lambda x} dx \right| \leq
    \frac{\varepsilon}{2} + \frac{\varepsilon}{2} = \varepsilon
  \end{gather*}
  Итак, \eqref{lem461:lim1} доказано. \\
  Отделяя действительные и мнимые части, получаем \eqref{lem461:lim2},
  \eqref{lem461:lim3}
\end{proof}

\begin{definition}
  Говорят, что функция $f$, заданная в проколотой окрестности точки $x$,
  удовлетворяет условиям Дини, если при $x \in \mathbb{R}$ выполняется:
  \begin{enumerate}
    \item В точке $x$ существуют оба предела:
      \begin{gather*}
        f(x-0) = \lim\limits_{t \to +0} f(x - t) \\
        f(x+0) = \lim\limits_{t \to +0} f(x + t)
      \end{gather*}
    \item $\int\limits_0^{\delta} \frac{f(x - t) - f(x - 0)}{t} dt \ $,
      $\int\limits_0^{\delta} \frac{f(x + t) - f(x + 0)}{t} dt$ сходятся
      абсолютно на $[0, \delta], \ \forall \delta > 0$
  \end{enumerate}
\end{definition}

\begin{theorem}
  $f$ --- $2\pi$-периодическая функция $f \in \mathcal{R}^1[-\pi, \pi], \ x \in
  \mathbb{R}$. \\
  \begin{gather}
    \sum\limits_{\infty} C_k(f) e^{ikx} = \frac{f(x-0) + f(x+0)}{2}
    \label{th461:eq1}
  \end{gather}
\end{theorem}

\begin{proof}
  $S_n(f, x) = \frac{1}{2\pi} \int\limits_{-\pi}^\pi f(x-t) D_n(t) dt =
  \frac{1}{2\pi} \int\limits_{-\pi}^0 f(x-t) D_n(t) dt + \frac{1}{2\pi}
  \int\limits_{0}^\pi f(x-t) D_n(t) dt = \frac{1}{2\pi} \int\limits_{0}^\pi
  f(x+t) D_n(t) dt + \frac{1}{2\pi} \int\limits_{0}^\pi f(x-t) D_n(t) dt =
  \frac{1}{2\pi} \int\limits_{0}^\pi (f(x - t) + f(x + t)) D_n(t) dt = \\
  \frac{1}{\pi} \int\limits_0^\pi \frac{f(x-0) + f(x+0)}{2} D_n(t) dt +
  \frac{1}{\pi} \int\limits_0^\pi \left(\frac{f(x-t) - f(x-0)}{2} +
  \frac{f(x+t) - f(x+0)}{2}\right) D_n(t) dt = \\
  \frac{f(x-0) + f(x+0)}{2} + \frac{1}{\pi} \int\limits_0^\pi \left(
  \frac{f(x-t) - f(x-0)}{2\sin \frac{t}{2}} + \frac{f(x+t) - f(x+0)}{2\sin
  \frac{t}{2}} \right) \sin((n+\frac{1}{2}) t) \ dt
  \label{th461:eqs}$ \\
  Поскольку, $2 \sin \frac{t}{2} \sim t, \ t \to 0$, то из условий Дини
  следует, что $g_x(t) \in \mathcal{R}^1[0, \pi]$ абсолютно интегрируема. На основании
  леммы Римана:
  \begin{gather*}
    \int\limits_0^\pi g_x(t) \sin ((n + \frac{1}{2})t) dt \to 0, \text{при} \ n
    \to \infty
  \end{gather*}
  Отсюда из \eqref{th461:eqs} следует \eqref{th461:eq1}
\end{proof}

\begin{consequence}
  Пусть $f$ --- ограниченная функция с периодом $2\pi$, имеющая разрыв первого
  рода и пусть имеет левые и правые производные. Тогда ряд Фурье сходится
  всюду, а его сумма в точке разрыва непрерывна и равна: $f(x) =
  \frac{1}{2}(f(x+0) + f(x-0))$
\end{consequence}

\section{Гладкость функции и скорость убывания коэффициентов Фурье}
\begin{definition}
  Функцию $f$ называют кусочно диффиренцируемой, если существует ее разбиение
  $a = x_0 < x_1 < \dots < x_n = b$, такое что на каждом интервале $(x_{j-1},
  x_j) f$ --- диффиренцируема, а в точках $x_{j-1}, x_j$ существуют конечные
  значения $f(x_{j-1}+0), f(x_j-0), f_+'(x_{j-1}), f_-'(x_j)$, при $\forall j =
  1, \dots, n, \ f' \in Q[a,b]$ \\
  Через $C_{2\pi}^{(k)} (k = 0, 1, \dots)$ обозначим класс $2\pi$-периодических
  (комплексно-значимых) функций, имеющих на $\mathbb{R} \ k$-ую непрерывную
  производную. $f^{(k)}, \ (f^{(k)} \in C_{2\pi} = C_{2\pi}^{(0)})$.
\end{definition}

\begin{lemma}[о диффиренцировании ряда Фурье]
  Если $f$ --- непрерывна, $f \in C_{2\pi}$ и $f$ кусочно диффиренцируема на
  $[-\pi, \pi]$, то $f' \sim \sum\limits_{-\infty}^{+\infty} C_k(f') e^{ikx}$
  может быть получен формальным диффиренцированием ряда Фурье $f \sim
  \sum\limits_{-\infty}^{+\infty} C_k(f) e^{ikx}$ самой функции, то есть:
  \begin{gather}
    C_k(f') = ikC_k(f), \ k \in \mathbb{Z}
    \label{lem471:eq1}
  \end{gather}
\end{lemma}

\begin{proof}
  Интегрированием по частям находим:
  \begin{gather*}
    C_k(f') = \frac{1}{2\pi} \int\limits_{-\pi}^\pi f'(x) e^{-ikx} dx =
    \left. \frac{1}{2\pi} f(x) e^{-ikx} \right|_{-\pi}^\pi + \frac{ik}{2\pi}
    \int\limits_{-\pi}^\pi f(x) e^{-ikx} dx = ikC_k(f)
  \end{gather*}
\end{proof}

\begin{theorem}
  Пусть $f \in C_{2\pi}^{(m-1)}, \ (m \in \mathbb{N}), \ f^{(m-1)}$ ---
  кусочно-диффиренцируема \\ на $[-\pi, \pi]$. Тогда:
  \begin{gather}
    C_k(f^{(m)}) = (ik)^m C_k(f), \ k \in \mathbb{Z}
    \label{th471:eq1} \\
    |C_{\pm k}(f)| = \frac{\gamma_k}{k^m} =
    \bar{\bar{o}}\left(\frac{1}{k^m}\right), \ k
    \to \infty, k \in \mathbb{N}
    \label{th471:eq2}
  \end{gather}
  Причем $\sum\limits_{k = 1}^{\infty} \gamma_k^2 < \infty, \ \gamma_k \downarrow
  0, \ \gamma_k = \bar{\bar{o}}\left(\frac{1}{k}\right)$.
\end{theorem}

\begin{proof}
  Соотношение \eqref{th471:eq1} получается в результате $m$-кратного
  использования \eqref{lem471:eq1}. \\
  $C_k(f^{(m)}) = ik C_k(f^{(m-1)}) = \dots = (ik)^m C_k(f), \ k \in
  \mathbb{Z}$. \\
  Полагаю $\gamma_k = |C_k(f^{(m)})|$, с учетом неравенства Бесселя:
  \begin{gather*}
    \sum\limits_{-\infty}^{+\infty} \gamma_k^2 \leq \frac{1}{2\pi}
  \end{gather*}
  Из \eqref{th471:eq1} получаем \eqref{th471:eq2}.
\end{proof}

\begin{theorem}
  $f \in C_{2\pi}^{(m-1)}, \ m \in \mathbb{N}, \ f^{(m-1)}$ ---
  кусочно-диффиренцируема на $[-\pi, \pi]$, тогда ряд Фурье сходится абсолютно
  и равномерно на $\mathbb{R}$, причем
  \begin{gather}
    \max\limits_{x \in \mathbb{R}} |f(x) - S_n(f,x)| =
    \bar{\bar{o}}\left(\frac{1}{n^{m - 1/2}}\right), \ n \to \infty
    \label{th472:eq1}
  \end{gather}
\end{theorem}

\begin{proof}
  Поскольку $f \in C_{2\pi}$ и удовлетворяет условиям Дини в любой точке, то
  \begin{gather*}
    f(x) = \lim\limits_{n \to \infty} S_n(f, x), \ \forall x \in \mathbb{R}
  \end{gather*}
  Поэтому, с учетом \eqref{th471:eq2} имеем:
  \begin{gather}
    |f(x) - S_n(f, x)| = \left|\sum\limits_{|k| \geq n + 1} C_k(f)
    e^{ikx}\right| \leq \left|\sum\limits_{|k| \geq n + 1}
    |C_k(f)e^{ikx}|\right| = \\
    2 \sum\limits_{k = n + 1}^{\infty} \frac{\gamma_k}{k^m} \leq
    2 \left(\sum\limits_{k = n + 1}^{\infty} \gamma_k^2 \right)^{\frac{1}{2}}
    \left(\sum\limits_{k = n + 1}^{\infty}  \frac{1}{k^{2m}}\right)^{\frac{1}{2}}
    \label{th472:uneq1}
  \end{gather}
  (В силу неравенства Коши-Буняковского). \\
  Так как
  \begin{gather*}
    \left(\sum\limits_{k = n + 1}^{\infty}
    \frac{1}{k^{2m}}\right)^{\frac{1}{2}} \leq \left(\int\limits_{n+1}^{\infty}
    \frac{dx}{x^{2m}} \right)^\frac{1}{2} = \frac{1}{\sqrt{2m-1}} \cdot
    \frac{1}{n^{m-1/2}}
  \end{gather*}
  То из \eqref{th472:uneq1} получаем \eqref{th472:eq1}
\end{proof}

\begin{remark}
  Поскольку $a_k(f) = C_k(f) + C_{-k}(f), \ b_k(f) = i(C_k(f) - C_{-k}(f))$, то
  из \eqref{th471:eq2} следует, что
  \begin{gather*}
    |a_k(f)| = \frac{\alpha_k}{k^m}, \ |b_k(f)| = \frac{\beta_k}{k^m}, \ k \in
    \mathbb{N}, \ \sum \alpha_k^2 < \infty, \ \sum \beta_k^2 < \infty
  \end{gather*}
\end{remark}

