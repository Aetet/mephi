\documentclass[12pt]{article}
%\usepackage{ucs}
\usepackage[utf8x]{inputenc} % Включаем поддержку UTF8
\usepackage[russian]{babel}  % Включаем пакет для поддержки русского языка
\usepackage{amsmath}
\usepackage{amssymb}

\title{Дискретная математика}

\date{}
\author{abcdw}

\begin{document}
    \maketitle
    $$\frac{\mbox{Число искомых событий}}{\mbox{Число всех событий}}$$
    
    Есть урна, в ней черные и белые шары, a - белых, b - черных. \newline
    Достаем один шарик. $\frac{a}{a+b}$ \newline
    Два шара одновременно, какова вероятность, что они оба белые? $\frac{C_a^2}{C_{a+b}^2}$ \newline
    Пять шаров, какова вероятность, что два белых, три черных. $\frac{C_a^2 \times C_b^3}{C_{a+b}^5}$ \newline
    Две урны, выбираю по одному шару. $\frac{a}{a+b} \times \frac{c}{c+d}$ \newline

    Лифт с семью пассажирами может остановиться на 10 этажах, найти вероятность события, что никакие два пассажира не выйдут на одном этаже. $\frac{10\times 9 \times 8 \times 7 \times 6 \times 5 \times 4}{10^7}$ \newline

    В США 50 штатов, от каждого штата выбираем 2ух сенаторов и того получаем 100 сенаторов, терерь выбираем 50 человек, получили комитет, найти вероятность, что в этом комитете представлены все штаты. $\frac{2^{50}}{C_{100}^{50}}$ \newline
    Что данный конкретный штат представлен в этом комитете. $\frac{C_2^1\times C_{100 - 2}^{50 -1} + C_2^2 \times {C_{100 - 2}^{50 - 2}}}{C_{100}^{50}}$ \newline

    У нас $8 \times  4 = 32$ карт, две откладываю в сторону, остальные раздаю $10 \times 3$ \newline
    Один человек обнаружил, что у него 6 карт одной масти, какова вероятность, что в прикупе еще 2 карты той же масти. $\frac{1}{C_{32 - 10}^2}$ \newline

    a деталей, b - брак, выбираем c деталей, в которых в точности d - брак. $\frac{C_b^{d} \times C_{a-b}^{c-d}}{C_a^c}$\newline

\end{document}
