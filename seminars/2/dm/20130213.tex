\documentclass[12pt]{article}
%\usepackage{ucs}
\usepackage[utf8x]{inputenc} % Включаем поддержку UTF8
\usepackage[russian]{babel}  % Включаем пакет для поддержки русского языка
\usepackage{amsmath}
\usepackage{amssymb}

\title{Дискретная математика}

\date{}
\author{abcdw}

\begin{document}
    \maketitle

    Умножение, если события независимы или зависимость известна. \newline

    Квадратный стол. Север при западе - зафиксированная зависимость.
    Сколькими способами на одно место я могу посадить 4 разных человека. \newline

    Разбили, переформулировали - все ок. \newline

    В первенстве по футболу играет 10 разных команд. 1 место - 1 команда. Сколькими способами я могу определить 1, 2, 3 место. $P_{10}^{3}$ \newline
    Сколькими способами я могу выбрать трех призеров. $C_{10}^{3}$ \newline

    Порядок важен - перестановки. Порядок не важен - сочетания. \newline

    У одного студента 6 книгу, у другого 7. Хотят поменяться, сколькими способами они могут сделать это. \newline
    Две любые книги одного студента меняются на две другого. \newline

    У студента есть 2 яблока, 3 грушы, 4 апельсина. \newline
    7 мужчин, 4 женщины. Нужно выбрать 6 человек, не менее 2ух женщин. $C_7^4 * C_4^2 + C_7^3 * C_4^3 + C_7^2 * C_4^4$ \newline

    На собрании должно выступить 4 человека, но B всегда должен выступает после A. \newline
    Среди всех студентов 1ого курса МИФИ, в обязательном порядке найдутся по крайней мере два человека с одинаковыми инициалами. \newline
    В некотором царстве в некотором государстве не было двух людей с одинаковым набором зубов. Какова максимальная численность населения этого государства? \newline
    Сколькими способами можно одеть 5 колец на пальцы одной руки. \newline
\end{document}
