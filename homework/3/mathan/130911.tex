%HSE Template
\documentclass[a4paper,12pt]{article}

\usepackage[unicode,colorlinks=true,linkcolor=blue]{hyperref}
\usepackage{amsmath,amssymb}
\usepackage[utf8]{inputenc}
\usepackage[T2A]{fontenc}
\usepackage[russian]{babel}
\usepackage{graphicx}
\usepackage[margin=1in]{geometry}
\usepackage{fancyhdr}

\pagestyle{fancy}
\makeatletter
\fancyhead[L]{\footnotesize К3-682, 2013/14, <<Математический анализ>>}
\fancyfoot[L]{\footnotesize \@author}
\fancyfoot[R]{\thepage}
\fancyfoot[C]{}

\renewcommand{\maketitle}{%
\noindent{\bfseries\scshape\large\@title\ \mdseries\upshape(\@date)}\par
\noindent {\large\itshape\@author}
\vskip 2ex}
\makeatother

\newenvironment{problem}[1]{\par\bigskip\noindent\textbf{Решение задачи #1.}
  \enskip\ignorespaces}{}

\title{Решение домашнего задания} % Fill the number of the homework
\author{Тропин А. Г. \\
  e-mail: \href{mailto:andrewtropin@gmail.com}{andrewtropin@gmail.com} \\
  github: \href{http://github.com/abcdw/mephi}{abcdw/mephi}}
\date{11 сентября 2013 г.} % Fill the date

\begin{document}
  \maketitle
  \thispagestyle{fancy}

  \begin{problem}{3475}

    $x^2 \frac{\partial z}{\partial x} + y^2 \frac{\partial z}{\partial y} =
    z^2$, если $u = x$, $v = \frac{1}{y} - \frac{1}{x}$,
    $w = \frac{1}{z} - \frac{1}{x}$. \\
    $\frac{\partial w}{\partial x} =
    - \frac{1}{z^2} \frac{\partial z}{\partial y} - 1$,
    $\frac{\partial w}{\partial y} = - \frac{1}{z^2}
    \frac{\partial z}{\partial y}$ \\
    $\frac{\partial z}{\partial x} = -\frac{\partial w}{\partial x} z^2
    + \frac{z^2}{x^2}$,
    $\frac{\partial z}{\partial y} = -\frac{\partial w}{\partial y} z^2$ \\
    $z^2 - \frac{\partial w}{\partial x}x^2 z^2
    - \frac{\partial w}{\partial y} y^2 z^2 = z^2$ \\
    $1 - \frac{\partial w}{\partial x}x^2
    - \frac{\partial w}{\partial y} y^2 = 1$ \\
    $\frac{\partial w}{\partial x}x^2
    - \frac{\partial w}{\partial y} y^2 = 0$ \\
    $\frac{\partial w}{\partial x} = 
    \frac{\partial w}{\partial u} \frac{\partial u}{\partial x}
    + \frac{\partial w}{\partial v} \frac{\partial v}{\partial x} 
    = \frac{\partial w}{\partial u} 
    + \frac{\partial w}{\partial v} \frac{1}{x^2}$ \\
    $\frac{\partial w}{\partial y} 
    = \frac{\partial w}{\partial u} \frac{\partial u}{\partial y}
    + \frac{\partial w}{\partial v} \frac{\partial v}{\partial y}
    = - \frac{1}{y^2} \frac{\partial w}{\partial v}$ \\
    $x^2 \frac{\partial w}{\partial u}
    + \frac{\partial w}{\partial v} \frac{x^2}{x^2} 
    - \frac{\partial w}{\partial v} \frac{y^2}{y^2} = 0$,
    $x^2 \frac{\partial w}{\partial u} = 0$ \\

    Ответ: $\frac{\partial w}{\partial u} = 0$

  \end{problem}

  \begin{problem}{2550}

    $S_n = \frac{1}{1 * 4} + \frac{1}{4 * 7} + \dots + \frac{1}{(3n-2)(3n+1)}$\\
    $\frac{1}{(3n-2)(3n+1)} = \frac{A}{3n-2} + \frac{B}{3n + 1}$ \\
    $A - 2B = 1, A = -B \Rightarrow A = \frac{1}{3}, B = -\frac{1}{3}$ \\
    $a_n = \frac{1}{3} \left( \frac{1}{3n-2} 
    - \frac{1}{3n + 1}\right)$ \\
    $S_n = \frac{1}{3} \left( 1 - \frac{1}{3n+1} \right)$
    
    Ответ: $S = \frac{1}{3}$
  \end{problem}

  \begin{problem}{2552}

    $\sum\limits_{n=1}^{\infty} (\sqrt{n+2} - 2\sqrt{n+1} + \sqrt{n})$ \\
    $S_n = 1 - \sqrt{2} - \sqrt{n+1} + \sqrt{n+2}$

    Ответ: $S = 1 - \sqrt{2}$
  \end{problem}
  
  \begin{problem}{2560}

    $\sum\limits_{n = 1}^{\infty} \frac{1}{1000n+1}$ \\
    $a_n = \frac{1}{1000n+1} \sim \frac{1}{n^p}(p = 1)$

    Ответ: по признаку сравнения ряд рассходится.

  \end{problem}

  \begin{problem}{2561}

    $a_n = \frac{n}{2n - 1}$ \\
    $\lim\limits_{n \to \infty} a_n = \frac{1}{2} \not = 0$

    Ответ: не выполняется необходимое условие сходимости.
  \end{problem}

  \begin{problem}{2562}

    $a_n = \frac{1}{(2n - 1)^2}$ \\
    $a_n \sim \frac{1}{n^p} (p = 2)$

    Ответ: по признаку сравнения ряд сходится.
  \end{problem}

  \begin{problem}{2563}

    $a_n = \frac{1}{n\sqrt{n+1}}$ \\
    $a_n \sim \frac{1}{n^p} (p = \frac{3}{2})$

    Ответ: по признаку сравнения ряд сходится.
  \end{problem}

  \begin{problem}{2564}

    $a_n = \frac{1}{\sqrt{(2n - 1)(2n + 1)}}$ \\
    $a_n \sim \frac{1}{n^p} (p = 1)$

    Ответ: по признаку сравнения ряд рассходится.
  \end{problem}

  \begin{problem}{2578}

    $a_n = \frac{1000^n}{n!}$ \\
    $\lim\limits_{n \to \infty} \frac{1000^{(n+1)}}{(n+1)!} \frac{n!}{1000^n}
    = \lim\limits_{n \to \infty} \frac{1000}{n + 1} = 0 < 1$

    Ответ: по признаку Даламбера ряд сходится.
  \end{problem}

  \begin{problem}{2579}

    $a_n = \frac{(n!)^2}{(2n)!}$ \\
    $\lim\limits_{n \to \infty} \frac{((n+1)!)^2}{(2(n+1))!}
    \frac{(2n)!}{(n!)^2} = \lim\limits_{n \to \infty} 
    \frac{(n+1)(n+1)}{(2n+1)(2n+2)} = \frac{1}{2} \frac{1}{2} = \frac{1}{4} < 1$

    Ответ: по признаку Даламбера ряд сходится.
  \end{problem}

  \begin{problem}{2580}

    $a_n = \frac{n!}{n^n}$ \\
    $n! \approx \sqrt{2\pi n} \left( \frac{n}{e} \right) ^n$ 
    \footnote{формула Муавра — Стирлинга} \\
    $\lim\limits_{n \to \infty} \sqrt[n]{\frac{n!}{n^n}}
    = \lim\limits_{n \to \infty} 
    \sqrt[n]{\frac{\sqrt{2\pi n} \left( \frac{n}{e} \right) ^n}{n^n}}
    = \frac{1}{e} < 1$

    Ответ: по признаку Коши ряд сходится.
  \end{problem}
  
  \begin{problem}{2581(а)}
    
    $a_n = \frac{2^nn!}{n^n}$ \\
    $n! \approx \sqrt{2\pi n} \left( \frac{n}{e} \right) ^n$ \\
    $\lim\limits_{n \to \infty} \sqrt[n]{\frac{2^nn!}{n^n}}
    = \lim\limits_{n \to \infty} 
    \sqrt[n]{\frac{2^n\sqrt{2\pi n} \left( \frac{n}{e} \right) ^n}{n^n}}
    = \frac{2}{e} < 1$

    Ответ: по признаку Коши ряд сходится.
  \end{problem}

  \begin{problem}{2581(б)}
    
    $a_n = \frac{3^nn!}{n^n}$ \\
    $n! \approx \sqrt{2\pi n} \left( \frac{n}{e} \right) ^n$ \\
    $\lim\limits_{n \to \infty} \sqrt[n]{\frac{3^nn!}{n^n}}
    = \lim\limits_{n \to \infty} 
    \sqrt[n]{\frac{3^n\sqrt{2\pi n} \left( \frac{n}{e} \right) ^n}{n^n}}
    = \frac{3}{e} > 1$

    Ответ: по признаку Коши ряд рассходится.
  \end{problem}

  \begin{problem}{2582}

    $a_n = \frac{(n!)^2}{2^{n^2}}$ \\
    $\lim\limits_{n \to \infty} \frac{(n+1)!^2}{2^{n^2 + 2n + 1}}
    \frac{2^{n^2}}{(n!)^2}
    = \lim\limits_{n \to \infty} \frac{(n+1)(n+1)}{2^{2n+1}} = 0 < 1$

    Ответ: по признаку Даламбера ряд рассходится.
  \end{problem}
\end{document}
